
\begin{abstract}
Model Driven Engineering (MDE) is a software development methodology that
relies  on developing domain models that represent knowledge concepts and
activities that belong to the application domain. When applied in software development, MDE aims
to support automatic generation of source code from the domain models, which
also provides a means for keeping design models and implementation synchronised.

One of the modeling languages that can be used in MDE is Colored Petri Nets
(CPN). It is a type of high-level Petri Net, suited for describing,
analysing and validating systems that consist of communication, synchronisation,
and resource sharing between concurrently executing components. A CPN model can
accurately model many types of software systems, but cannot directly be used
to generate a software implementation. Research is being conducted to develop an
approach for annotating CPN models of communication protocols to enable source
code generation.

This thesis has resulted in a prototype application for annotating CPN models
with code generation pragmatics. The prototype builds on the ePNK framework,
which uses the Eclipse Modeling Framework (EMF) to provide an extensible
platform for working with CPN models.
The prototype is desgned as a plugin for Eclipse. It can import
models created by CPN Tools and lets the user annotate the models with code
generation pragmatics. The prototype has been evaluated by applying it to a set
of protocol CPN models. We show that CPN models annotated with code generation
pragmatics are a viable method for use in model-driven development of protocol
software.
We also give a detailed case study of modeling the WebSocket protocol, including
verification and analysis of the CPN model using state space exploration, and
annotation with code generation pramatics.

\end{abstract}
