%\chapter*{Abstract}
%\addcontentsline{toc}{chapter}{Abstract}

\begin{abstract}
Model Driven Engineering (MDE) is a software development methodology that
focuses on developing domain models that represent knowledge and activities that
belong to the application domain. When applied in software development, MDE aims
to provide automatic generation of source code from the domain models, which
also functions as a mechanic for keeping design and implementation synchronised.

One of the modeling languages that can be used in MDE is Colored Petri Nets
(CPN). It is a type of high-level Petri Net, whic is useful for for describing,
analysing and validating systems that consist of communication, synchronisation,
and resource sharing between concurrently executing components. 

A CPN model can accurately model many types of software systems, but cannot
directly be used to generate a software implementation. Research is being
conducted to define a framework/method? for annotating CPN models of
communication protocols to enable source code generation. 

This thesis has resulted in a prototype application for annotating CPN models
with code generation pragmatics. The prototype is developed as an Eclipse
plugin. It builds on the ePNK framework, which uses the Eclipse Modeling
Framework (EMF) to provide an extensible platform for working with CPN models.
The prototype can import models created by CPN Tools. 

The prototype is evaluated by applying it to a set of protocol CPN models. We
show that CPN models annotated with code generation pragmatics are a viable
method for use in MDE of protocol software. We also give a detailed case study
of modeling the WebSocket protocol, including verification and analysis of the
CPN model using State Space Generation, and annotation with code generation
pramatics.

\end{abstract}
