\chapter{Analysis and Design}
\label{chap:analysis}

\section{Requirements} 
Before we describe the plugin itself, the requirements need to be detailed.
There are four main parts to this:

\begin{itemize}
	\item Loading models created in CPN Tools.

	\item Annotate model with pragmatics. 
		
	\item Load sets/classes of pragmatics to add to model
	
	\item Create model-specific pragmatics during model construction

\end{itemize}

\section{Test cases}


Simple protocol (TCP)

Kao-chow authentication protocol

The Edge Router Discovery Protocol (ERDP) for mobile ad-hoc networks

The WebSocket Protocol

\section{Defining Pragmatics}
TODO: Detaljert info om pragmatikker og hva de er

A Pragmatic is an annotation on a model element that describes how it should be
translated into code.

Pragmatics come in three types: General, domain-specific and model-specific.

In the context of network protocols, general pragmatics include concepts
like channels, principals, and operations like opening or closing a connection,
and sending or receiving data. 

Domain-specific pragmatics are things that apply to all (or many) protocols
within a domain. An example is security protocols, where examples of domain
specific operations are encryption, decryption, and nonce generation.

Model-specific pragmatics apply only to the model in which they are defined.
Examples in the context of the WebSocket Protocol are masking and unmasking of
frames, and integrity checking of received frames.

\section{The CPN Ontology with Pragmatics}
(Obsolete ettersom Kent er på vei bort fra SADL, skal skrives om til Manchester
syntax)

\section{The Annotated CPN model type for ePNK}

The first prototype we created was a single model that contained
attributes for CPN inscriptions, new classes for supporting pragmatics, as
well as logic for constraining the model according to the rules of Colored Petri
Nets. As the model grew, we realised it would be beneficial to separate the
model into two parts: One to define pure CPN models, and another for defining
pragmatics, extending from the first. This also adds the benefit that the pure
CPN model can be used for other applications.

\fig{CPNDefinition.pdf}{CPN model type diagram}{model_diagram}
\fig{PragmaCPNDefinition.pdf}{Annotated CPN model type
diagram}{ann_model_diagram} 

A standard Petri Net model created in ePNK is initially
only defined by the PNML Core Model type, but can add a PetriNetType object to
extend it with the features of another Petri Net model type. Only one PetriNetType
object can exist in a Petri Net. Once such an object is added, ePNK will
dynamically load any related plugin(s), and the menus for adding new objects to
the model will include any new classes that the Petri Net Type defines.

\fig{plugin_manifest_1.png}{Plugin Manifest part 1}{plugin_manifest_1}
A custom Petri Net Type is made by creating an Eclipse Plugin project.
The plugin manifest needs to be edited to define this plugin as an
extension to the org.pnml.tools.epnk.pntd extension point of ePNK. A unique id
and a descriptive name should be provided. \figref{plugin_manifest_1} shows the
configuration for the CPN model type. However, before we can fully configure the
extension, we have to create our model and generate source code for it.

Next, an EMF model should be created. This model should inherit
the PNML Core Model from ePNK, or any other model that already does this (like
P/T-Net and HLPNG does). \com{Glossary for P/T-Net and HLPNG} The first thing
this new Model Type should define is a subclass of the PetriNetType class with
the name of the new Model Type, which in our case is CPN. This can be seen in
the top left corner of the diagram in \figref{model_diagram}. This class is what
will appear in the menu to add a Petri Net Type to a new model in ePNK.
\com{add fig?}


After creating this model, source code for interfaces and implementations of
the new class can be generated by EMF. \com{describe how} The source file for
the implementation of the PetriNetType subclass needs two minor modifications
to work with ePNK: The constructor must be made public (it is protected by
default), and the toString method must be implemented to conform to the
PetriNetType interface. This method should return a string that textually
represents the net type, usually simply a name.

\fig{plugin_manifest_2.png}{Plugin Manifest part 2}{plugin_manifest_2}
After generating the source code for the PetriNetType subclass, the extension
configuration can be completed. The full class name of the PetriNetType subclass
must be provided, to perform as a starting point for ePNK to include the model,
and a short description of the model type. 

In order to change the functionality of existing classes such as Place, Arc and
Page, they have to be subclassed. EMF does not support
merging of models, so it is not possible to define new properties or relations
directly on the referenced classes of the Core Model. Instead, ePNK will check
the new Petri Net Type model for subclasses with the same name, and load these
dynamically instead of the base classes. This can be seen in
\figref{plugin_manifest_1}, where Place, Arc, Transition and Page are all
subclassed from the classes referenced from the Core Model, with added
attributes needed to represent Coloured Petri Nets.

TODO: Constraints

TODO: Intro Pragma Net Type
These subclasses then have references to the Pragma class. The references are
configured to act as containment, meaning Pragma instances are created as
children of the related classes. TODO: Smooth overgang. As we can see in
\figref{add_child_pragma_menu}, Pragma objects are now available as children to Places.

\fig[0.6]{ePNK_AddChildMenu.png}{Adding Pragma as child to
Place}{add_child_pragma_menu}

The Pragma class inherits from Label, which lets it be represented visually as a
text label in the graphical editor of ePNK.

\section{Importing from CPN Tools}
Access/CPN is a framework that can parse CPN models saved by CPN Tools and
represent the model with EMF classes. Access/CPN has many additional features
related to the semantics of CPN, but only the model importer is interesting to
us.

The EMF model for CPN models that Access/CPN defines uses many of the same class
names as ePNK, making code difficult to write and read due to the need to use
fully qualified classpaths to avoid name collisions. Because of this, and the
fact that Access/CPN at the time was missing a few features like importing
graphics, we first considered creating a new parser based on the code in
Access/CPN. But Access/CPN has continually been improved by its author during
development of our plugin, and is now also capable of parsing graphics data. And
by depending on Access/CPN instead of writing our own parser, we can also
benefit from further updates more easily.

The conversion is very straightforward, the .cpn file is loaded with Access/CPN,
and the resulting model is then converted object by object to the ePNK CPN type.

\section{Creating annotations}
Labels on nodes, arcs and inscriptions. 
Choose from list. Possibly write freehand with content assist. Validation, with
problem markers (Eclipse feature).

\section{Choosing Pragmatics Sets}
Where to store? Model, Project, Plugin
Model pros:
	Will need anyway for model-specific pragmatics
	Could be associated with net type as a property (like HLPNG does) or as a sub
	node somewhere
	Have URI string and version to check.
cons:
	Keeping base pragmatics up to date a problem
Namespace-based? Already required for ontology


	\subsection{Creating custom pragmatics}
	Dynamically supported in content assist 
	If ontology-based, use SADL editor
	På sikt eget verktøy
		Hva kan det settes på, hvilke attributter har det.
		Oversette til ontologi

