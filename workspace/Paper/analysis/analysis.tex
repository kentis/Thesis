\chapter{Analysis, Design and Implementation}
\label{chap:analysis}


After picking the technologies described in the previous chapter, since many of
the componets have Eclipse in common, it was an easy decision to develop
\thename{} as an Eclipse plugin. This also let us centralise all our development
in Eclipse.

Our analysis of available software solutions and platforms showed that the basic
primitives that are required to develop the desired code generation pragmatic
framework are available as part of the Eclipse eco system. It is therefore
natural to develop \thename{} as an Eclipse Plugin and base all software
develpoment in the context of this thesis on the Eclipse Platform.


\section{The CPN Ontology with Pragmatics}
As mentioned in section \ref{sec:ontologies}, Pragmatics are defined and modeled
as ontologies, using OWL 2 Web Ontology Language.  

There exists two ontologies that function as a base for pragmatics: One that
defines Coloured Petri Nets, and one that defines basic classes for pragmatics

\begin{lstlisting}[label=lst:cpn.owl,caption=The CPN Ontology] 
Prefix(:=<http://hib.no/ontologypetrinets/cpn/>)
Prefix(xsd:=<http://www.w3.org/2001/XMLSchema#>)
Prefix(owl:=<http://www.w3.org/2002/07/owl#>)
 
Ontology(<http://hib.no/ontologypetrinets/cpn/>
 

Declaration(Class(:ColoredPetriNet))
SubClassOf(:ColoredPetriNet :HighLevelPetriNet)
SubClassOf(:HighLevelPetriNet :PetriNet)

Declaration(Class(:Page))

Declaration(Class(:Node))
SubClassOf(:Node ObjectMaxCardinality(1 :page))
SubClassOf(:Node DataMaxCardinality(1 :name))

Declaration(Class(:Place))
SubClassOf(:Place :Node)

Declaration(Class(:Transition))
SubClassOf(:Transition :Node)

Declaration(Class(:Arc))
SubClassOf(:Arc ObjectMaxCardinality(1 :dest))
SubClassOf(:Arc ObjectMaxCardinality(1 :source))
SubClassOf(:Arc DataMaxCardinality(1 :inscription))

Declaration(Class(:NormalTransition))
SubClassOf(:NormalTransition :Transition)
SubClassOf(:NormalTransition DataMaxCardinality(1 :codeSegment))
SubClassOf(:NormalTransition DataMaxCardinality(1 :guard))
SubClassOf(:NormalTransition DataMaxCardinality(1 :timing))
SubClassOf(:Place DataMaxCardinality(1 :colorType))
SubClassOf(:Place DataMaxCardinality(1 :initialMarking))

Declaration(Class(:Port))
Declaration(Class(:Socket))
SubClassOf(:Socket :Place)
SubClassOf(:Port :Place)
SubClassOf(:Socket ObjectMaxCardinality(1 :direction))
SubClassOf(:Socket ObjectMaxCardinality(1 :port))
Declaration(Class(:SocketDirection))
EquivalentClasses(:SocketDirection ObjectOneOf(:Out :InOut :In))

Declaration(Class(:SubstiutionTransition))
SubClassOf(:SubstiutionTransition :Transition)
SubClassOf(:SubstiutionTransition ObjectMaxCardinality(1 :page))

DisjointClasses(:Page :Place :Transition :Arc)
DisjointClasses(:NormalTransition :SubstiutionTransition)

Declaration(ObjectProperty(:arcs))
ObjectPropertyDomain(:arcs :Page)
ObjectPropertyRange(:arcs :Arc)
Declaration(ObjectProperty(:dest))
ObjectPropertyDomain(:dest :Arc)
ObjectPropertyRange(:dest :Node)
Declaration(ObjectProperty(:direction))
ObjectPropertyDomain(:direction :Socket)
ObjectPropertyRange(:direction :SocketDirection)
Declaration(ObjectProperty(:nodes))
InverseObjectProperties(:page :nodes)
ObjectPropertyDomain(:nodes :Page)
ObjectPropertyRange(:nodes :Node)
Declaration(ObjectProperty(:page))
InverseObjectProperties(:page :nodes)
ObjectPropertyDomain(:page ObjectUnionOf(:Node :PetriNet :SubstiutionTransition))
ObjectPropertyRange(:page :Page)
Declaration(ObjectProperty(:port))
ObjectPropertyDomain(:port :Socket)
ObjectPropertyRange(:port :Port)
Declaration(ObjectProperty(:socket))
ObjectPropertyDomain(:socket :Port)
ObjectPropertyRange(:socket :Socket)
Declaration(ObjectProperty(:source))
ObjectPropertyDomain(:source :Arc)
ObjectPropertyRange(:source :Node)
Declaration(DataProperty(:codeSegment))
DataPropertyDomain(:codeSegment :NormalTransition)
DataPropertyRange(:codeSegment xsd:string)
Declaration(DataProperty(:colorType))
DataPropertyDomain(:colorType :Place)
DataPropertyRange(:colorType xsd:string)
Declaration(DataProperty(:guard))
DataPropertyDomain(:guard :NormalTransition)
DataPropertyRange(:guard xsd:string)
Declaration(DataProperty(:initialMarking))
DataPropertyDomain(:initialMarking :Place)
DataPropertyRange(:initialMarking xsd:string)
Declaration(DataProperty(:inscription))
DataPropertyDomain(:inscription :Arc)
DataPropertyRange(:inscription xsd:string)
Declaration(DataProperty(:name))
DataPropertyDomain(:name :Node)
DataPropertyRange(:name xsd:string)
Declaration(DataProperty(:timing))
DataPropertyDomain(:timing :NormalTransition)
DataPropertyRange(:timing xsd:int)
ClassAssertion(:SocketDirection :In)
ClassAssertion(:SocketDirection :InOut)
ClassAssertion(:SocketDirection :Out)
)

\end{lstlisting}

Every ontology must be declared with an Internationalized Resource
Identifier (IRI) . IRI is a generalization of the Uniform Resource Identifier
(URI) to allow all Unicode characters to be used. An IRI can point to an actual
resource on the internet, but is not required to do so.

An ontology document can import other ontologies by referring to their IRI. 

\section{The Annotated CPN model type for ePNK}
While designing the Type model for the Plugin, it was decided to separate it
into two parts: One to define a CPN Type, and one for defining an Annotated CPN
Type, extending from the first. This also adds the benefit that the pure CPN
Type can be used for other applications.

\com{Oppdater figurer! }
\fig{CPNDefinition.pdf}{CPN model type diagram}{cpn_model_diagram}
A custom Petri Net Type needs to be contained in an Eclipse Plugin project.
Such a project will contain configuration files that define the properties and
capabilities of the plugin. These files include MANIFEST.MF for declaring plugin
name, version, dependencies etc, plugin.xml for defining extensions, extension
points etc, and build.properties for defining build parameters. However, the
Eclipse Plugin IDE includes an editor for editing all three files in a
convenient user interface with guides and content assistance.

After creating a plugin project, an EMF model should be created. This is the
Type Model, and should inherit the PNML Core Model from ePNK, or any other model
that already does this (such as the P/T-Net or HLPNG Types). \com{Glossary for
P/T-Net and HLPNG}

The CPN model, shown in \figref{cpn_model_diagram}, defines the structure and
constraints of Coloured Petri Nets. The first thing this new Model Type should
define is a subclass of the PetriNetType class with the name of the new Model
Type, which in our case is CPN. This can be seen in the top left corner of the
diagram in \figref{cpn_model_diagram}. This class is what identifies the Type,
and is what will appear in the menu to let a user extend a model with the new
Type.
\com{add fig?}

After creating this model, EMF can generate source code for interfaces and
implementations of the new class. This is done by creating a ``genfile''
linked to the EMF model. The genfile can define metainfo such as the base
package of generated source files, and configuration parameters for the
individual classes. EMF can then generate different groups of code, but for a
ePNK Type we only need Model code and Edit code. the Model code includes
interfaces and corresponding basic implementations of the classes as well as
factories for instantiating them, while the Edit code contains classes for
presenting the Model code in an editor.

After generating the code for the Type Model, the source file for
the implementation of the PetriNetType subclass needs two minor modifications
to work with ePNK: The constructor must be made public (it is protected by
default), and the toString method must be implemented to conform to the
PetriNetType interface. This method should return a string that textually
represents the net type, usually simply its formal name.

\fig{plugin_manifest_1.png}{Plugin Manifest part 1}{plugin_manifest_1}
\fig{plugin_manifest_2.png}{Plugin Manifest part 2}{plugin_manifest_2}
Before ePNK will recognise the plugin and the Type Model, the plugin manifest
needs to be edited to define this plugin as an extension to the
org.pnml.tools.epnk.pntd extension point of ePNK. All that is needed to
configure this is supplyting a unique id, a descriptive name, and the fully
qualified classpath to th ePetriNetType subclass. \figref{plugin_manifest_1} and
\figref{plugin_manifest_2} shows the finished configuration for the CPN model
type. \com{Kan bytte ut figurene med XML-data istedet}

EMF does not support merging of models, meaning it is not possible to define new
properties or relations directly on the original classes of the Core Model.
Thus, in order to change the functionality of existing classes such as Place,
Arc and Page, they have to be subclassed. ePNK will use reflection to check a
Type Model for subclasses with the same name as classes in the Core Type, and
load these dynamically instead of the base classes.
This can be seen in \figref{cpn_model_diagram}, where Place, Arc, Transition and
Page are all subclassed from the classes referenced from the Core Model, with
added attributes needed to represent Coloured Petri Nets.

TODO: Constraints

The Annotated CPN model, shown in \figref{ann_model_diagram}, extends the CPN
model to enable annotation of model elements. It also handles references
to ontology documents that define sets of pragmatics.

\fig{PragmaCPNDefinition.pdf}{Annotated CPN model type
diagram}{ann_model_diagram} 

As with the CPN Type Model, we have subclassed the classes of model elements we
need to extend. All of these need to be extended to support the same feature:
Being annotated with pragmatics. Ideally, we would have extended a superclass of
these classes, for instance Node. But this will not work, since model merging is
not possible (as discussed earlier), and subclassing Node will not work, since
existing classes will not be subclasses of our new superclass and therefore not
inherit anything from it. BELIEVE ME I TRIED

We needed to devise a way to extend a group of classes without access to their
superclass, and ideally without copying code between the classes. The solution
was to define an interface that all of the target classes implement. This is the
OntologyMember interface. 

OntologyMember has a reference to the Pragma class. The reference is
configured to act as containment, meaning Pragma instances are created as
children of OntologyMember instances.

The Pragma class inherits from Label, which lets it be represented visually as a
text label in the graphical diagram editor of ePNK. As a consequence, it is
modeled as containment reference instead of a property. 

\section{Importing from CPN Tools}
Access/CPN is a framework that can parse CPN models saved by CPN Tools and
represent the model with EMF classes. Access/CPN has many additional features
related to the semantics of CPN, but only the model importer is of relevance for
the work in this thesis.

Access/CPN also uses EMF to represent models internally. The EMF model for CPN
models that Access/CPN defines uses many of the same class names as ePNK, which
makes it tedious to write and read code working between the two frameworks due
to the need to use fully qualified classpaths to avoid class name collisions.
Initially we planned to extract the parser source code from Access/CPN and
rewriting it to use the new ePNK CPN Type classes. This plan was later discarded
in favor of depending on the Access/CPN plugin, as Access/CPN has continually
been improved during development of \thename{}, and is now also capable of
parsing graphics data. And by depending on Access/CPN instead of writing our own
parser, we can also benefit from further updates more easily.

The conversion is straightforward, the .cpn file is loaded with Access/CPN,
and the resulting model is then converted object by object to the ePNK CPN type.

\section{Creating Pragmatic Annotations}
Labels on nodes, arcs and inscriptions. 
Choose from list. Possibly write freehand with content assist. Validation, with
problem markers (Eclipse feature).

\section{Choosing Pragmatics Sets}
Where to store? Model, Project, Plugin
Model pros:
	Will need anyway for model-specific pragmatics
	Could be associated with net type as a property (like HLPNG does) or as a sub
	node somewhere
	Have URI string and version to check.
cons:
	Keeping base pragmatics up to date a problem
Namespace-based? Already required for ontology


	\subsection{Creating custom pragmatics}
	Dynamically supported in content assist 
	If ontology-based, use plugin editor
	
A specialised tool for creating model specific pragmatics would have ben ideal.
Such a tool would give simple mechanics using GUI controls for specifying which
model elements a pragmatic can be attached to, and which parameters it has. The
Plugin Manifest editor is a good example of what we have in mind. However, this
could not be included for this thesis due to time constraints.
	
