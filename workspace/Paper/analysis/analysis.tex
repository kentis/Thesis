\chapter{Analysis, Design and Implementation}
\label{chap:analysis}

Our analysis of available software solutions and platforms in the previous
chapter showed that the basic primitives that are required to develop the
desired code generation pragmatic framework are available as part of the
Eclipse eco-system. It is therefore natural to develop \thename{} as an Eclipse Plugin
and base all software development in the context of this thesis on the Eclipse
Platform.

\section{The CPN Ontology with Pragmatics}

As mentioned in section \ref{sec:ontologies}, Pragmatics are defined and modelled
as ontologies, using the OWL 2 Web Ontology Language. We have created two
ontologies to act as a basis for specifying individual pragmatics:
One that defines Coloured Petri Nets, and one that defines basic classes for pragmatics.
The ontology defining CPN will be explained in several parts, with line
numbering continuing across parts.

\lstinputlisting[label=lst:cpn.owl:start,numbers=left,stepnumber=1,firstline=1,lastline=5,float,
caption=The CPN Ontology: Opening]
{D:/Documents/Master/Thesis/workspace/Ontologies/ontologies/cpn.fowl}

\lstref{cpn.owl:start} shows the first fragment of the CPN ontology. Every
ontology must be declared with |Ontology()| containing an Internationalized Resource
Identifier (IRI) and a series of declarations and axioms. IRI is a
generalization of the Uniform Resource Identifier (URI) scheme to allow all
Unicode characters to be used. An IRI can point to an actual resource on the Internet,
but is not required to do so. Each concept in an ontology is also identified
with an IRI, but to avoid repetition we define ``:`` as a prefix representing
this document's IRI (line 1). Prefixes can also be declared for other
ontologies, and some (like owl: and xsd:) are predefined (and stated on lines 2
and 3 for clarity). 

\lstinputlisting[label=lst:cpn.owl:base,numbers=left,stepnumber=1,firstline=7,lastline=42,float,firstnumber=7,
caption=The CPN Ontology: Base Classes]
{D:/Documents/Master/Thesis/workspace/Ontologies/ontologies/cpn.fowl} 

Inside the |Ontology()| declaration we write declarations and axioms to describe
the domain. \lstref{cpn.owl:base} shows the first such declarations. Whitespace
and indentation is syntactically irrelevant and only used for readability. We
first declare the class |:ColouredPetriNet| (line 7). We then declare the object
property |:pages|, and restrict its domain to |:ColouredPetriNet| and its range
to |:Page|\footnote{Note that both Access/CPN and ePNK use the term Page instead
of Module, so for convenience we will also use the term Page in the
implementation of \thename{}.} (lines 8-10). This means that there exists a
property |:pages|, that instances of |:ColouredPetriNet| can have this property,
and that the value of the property must be a |:Page|. The class |:Page| is
defined just below on line 12, but the order of declarations are not important,
as the whole document must be loaded before any processing or reasoning can be
performed on the ontology. Also note that there are two kinds of properties:
Object (which range must be a class) and Data (which range must be a datatype
primitive, like string or integer, defined in the xsd: ontology).

We then continue declaring all model elements in a similar manner. A |:Page| can
have many |:elements|. An |:Element| has a |:page|. Line 21 states
|SubClassOf(:Element ObjectMaxCardinality(1 :page))|, meaning an :Element can
have at most 1 |:page|.
Line 22 states |InverseObjectProperties(:page :elements)|, defining these
properties as inversely related. An example consequence of this is if we define
an individual |:Element| and specify its |:page|, it is implicitly known that
the |:Page| contains that individual in its |:elements|. An |:Element| can also
have an |:id|.

Lines 29-42 describe the class |:Node| as a subclass of :Element with a |:name|
property and |:in| and |:out| properties for connected |:Arc|s.

\lstinputlisting[label=lst:cpn.owl:place,numbers=left,stepnumber=1,firstline=44,lastline=75,float,firstnumber=44,
caption=The CPN Ontology: Place and Transition Classes]
{D:/Documents/Master/Thesis/workspace/Ontologies/ontologies/cpn.fowl} 

\lstref{cpn.owl:place} shows declarations relating to the |:Place| class, and
follows the same pattern as the earlier classes to describe the properties
|:sort|\footnote{The term Sort comes from Description Logic, which is 
sometimes used in relation to CPN inscriptions. Sort is used in Access/CPN
instead of the term Colour Set. We opted to do the same.} and |:initialMarking|.
It continues with declarations relating to the |:Transition| and
|:SubstitutionTransition| classes.

\lstinputlisting[label=lst:cpn.owl:arc,numbers=left,stepnumber=1,firstline=77,lastline=96,float,firstnumber=77,
caption=The CPN Ontology: Arc Class]
{D:/Documents/Master/Thesis/workspace/Ontologies/ontologies/cpn.fowl} 

\lstref{cpn.owl:arc} shows declarations relating to the |:Arc| class. The
properties |:dest| and |:source| are declared as inverse of |:in| and |:out|
(from |:Node|).

\lstinputlisting[label=lst:cpn.owl:disjoint,numbers=left,stepnumber=1,firstline=99,lastline=102,float,firstnumber=99,
caption=The CPN Ontology: Disjoint Classes]
{D:/Documents/Master/Thesis/workspace/Ontologies/ontologies/cpn.fowl} Finally,
\lstref{cpn.owl:disjoint} declare classes as disjoint. This is necessary, as
ontologies by default use open world assumptions, meaning that even though
something is not explicitly asserted, it may still be true. Thus, without these
declarations, it is possible for an individual to for instance be a |:Place| and
an |:Arc| at the same time. Thus, we need to ``close the world'' by defining
classes as disjoint, meaning a member of one class is never a member of any of
the other classes in the statement. To fully enforce Closed World Assumption
(CWA) further measures need to be taken, which we describe in Section
\ref{sec:cwa}.

\lstref{basic.owl} provides the base ontology for pragmatics. The purpose of the
base is to link CPN elements and pragmatic annotations. The specific pragmatics
that can be attached to CPN elements will be defined in subsequent ontologies.
 
\lstinputlisting[label=lst:basic.owl,numbers=left,stepnumber=1,
caption=The Basic Pragmatics Ontology]
{D:/Documents/Master/Thesis/workspace/Ontologies/ontologies/basic.fowl}

\lstref{basic.owl} illustrates that in an ontology document it is possible to
import other ontologies by referring to their IRI.  We describe later how this
is taken into account when loading ontology documents from projects and plugins.
In this ontology, we import the CPN Ontology (line 8) and also assign it the
prefix |cpn:| (line 5).

The ontology in \lstref{basic.owl} is much shorter than the CPN ontology,
declaring only one new class |:Pragmatic|, as well as its disjointness with
other classes and its |:belongsTo| property. It also declares the property
|:hasPragmatic| for |cpn:Element|, and declares the two properties as the
inverse of each other.

Next, we have the ontology for the Generic pragmatics that provide the
fundamental set of pragmatics for the protocol domain, shown in
\lstref{nppn.owl}. This is the first ontology that defines actual usable
pragmatics. Each of these pragmatics are defined using two declarations, where
the first asserts it as a subclass of |basic:Pragmatic|, and the second is a
form of property restriction. In the case of the first pragmatic, line 14 states
that a |:Principal| can only belong to a |cpn:SubstitutionTransition|. We will
give a brief explanation of the purpose and usage each of the pragmatics in 
Chapter \ref{chap:evaluation}, where we apply them to various CPN models. Note
that this is not an exhaustive list of all generic pragmatics, and is only meant
to be a working example.

\lstinputlisting[label=lst:nppn.owl,numbers=left,stepnumber=1,
caption=The Generic Pragmatics Ontology]
{D:/Documents/Master/Thesis/workspace/Ontologies/ontologies/generic/nppn.fowl}

The |:External| pragmatic (line 22 and 23) shows why ontologies are well suited
for defining pragmatics. It can belong to |:TransitionConnectedToId|, which is a
complex class defined through a chain of class assertions: We first define
|:IdPlace| on line 43 to be equivalent to the set of individuals that are a |cpn:Place|, and
has a |:hasPragmatic| property with a value that is an instance of |:Id|, and
has exactly one |:hasPragmatic| property associated. 

In the same manner we
further define |:ArcFromId| on line 50 as anything that is a |cpn:Arc| coming
from an |:IdPlace|. Finally on line 57, a |:TransitionConnectedToId| must have
an incoming |:ArcFromId|. All together this describes the circumstance where an
|:External| pragmatic can be placed, which is on any transition with an incoming
arc from a place annotated with an |:Id| pragmatic. We demonstrate the practical
effect of this when annotating models in Chapter \ref{chap:evaluation}.

\subsection{The Ontology Plugin}\label{sec:ontologiesPlugin}
The ontologies described above are not subject to change by the user of
\thename{}, and should be included automatically. We need a
way to conveniently provide them to \thename{}, as well as utilities for managing
them. As mentioned, the IRI identifying the ontology can point to an actual
resource on the Internet, but we want to avoid relying on an Internet
connection. 

To solve this we created a separate Eclipse plugin called Ontologies.
It contains the ontology documents, utility classes for listing all ontologies the
plugin provides, a custom PluginIRIMapper which can map the IRI of import
statements to the ontology documents available in the plugin, and an
OntologyLoader utility which creates an OWL API ontology manager loaded
with a PluginIRIMapper.

It also includes the OWL API and several reasoner engine library files, and
exports their packages to let other plugins use them. This also gives a central
point of upgrading libraries should it be desired in the future.

\section{The Annotated CPN model type for ePNK}
While designing the ePNK Petri Net Type model for \thename{}, it was decided to
separate it into two parts: One to define a CPN Type that corresponds to CPN
Tools including the various inscriptions and model constraints, and one
for defining an Annotated CPN Type, extending from the first and capturing how
pragmatics relate to elements of the CPN Type model.
This also adds the benefit that the CPN Type can be used in other
applications.

A custom Petri Net Type needs to be contained in an Eclipse Plugin project.
Such a project will contain configuration files that define the properties and
capabilities of the plugin. These files include MANIFEST.MF for declaring plugin
name, version and dependencies, plugin.xml for defining extensions and extension
points, and build.properties for defining build parameters. The Eclipse Plugin
IDE includes an editor for editing all three files in a convenient user
interface with guides and content assistance.

After creating a plugin project, an EMF model should be created. This is the
Type Model, and should inherit the PNML Core Model from ePNK, or any other model
that already does this (such as the P/T-Net or HLPNG Types\footnote{These
ePNK Types were briefly introduced in Section \ref{sec:epnk}}).
We will refer to the PNML Core Model as the Core Model.

\subsection{CPN Type Model}
The CPN Type model is shown in \figref{cpn_model_diagram}, and defines the
structure and constraints of Coloured Petri Nets. The first thing a new Type
model should define is a subclass of the PetriNetType class from the Core Model
(we will refer to this as the Core Type), with the name of the new Type, which in
our case is CPN. This can be seen in the top left corner of the diagram in
\figref{cpn_model_diagram}. This class is used to identify the Type, and is
what will appear in the menu to let a user extend a model with the new Type.

\fig[0.6]{CPNDefinition.pdf}{CPN model type diagram}{cpn_model_diagram}


EMF does not support merging of models, meaning it is not possible to define new
properties or relations directly on the original classes of the Core Model.
Thus, in order to change the functionality of existing classes such as Place,
Arc and Page, they have to be subclassed. ePNK uses reflection to check a
Type Model for subclasses with the same name as classes in the Core Type, and
load these dynamically instead of the base classes.
This can be seen in \figref{cpn_model_diagram}, where Place, Arc, Transition and
Page are all subclassed from the classes referenced from the Core Model, with
added relations to inscription classes.

Due to limitations in ePNK and the PNML specification, which we discuss in
Section \ref{sec:limitations}, we need to simplify the representation of substitution
transitions by defining an attribute on the Transition class to be able to
designate an instance of it as a substitution transition. We do not want to use
the RefTransition from the Core Type for this, as it is intended to reference
another transition, and since we are unable to change or remove this aspect of
RefTransition (for the same reasons discussed in the previous paragraph) it
could potentially confuse users.

To represent the different inscriptions of a CPN model, we define classes that
inherit the Label class from the Core Model. From a data modelling
perspective, it would make more sense to make inscriptions simple
string attributes of the classes they belong to, but the Label class includes
functionality required for displaying in the graphical diagram editor of ePNK.
The different inscription classes are identical implementation-wise, but have
different names to be distinguishable in the user editor context menu for adding
child elements.

\subsection{Code Generation}
After creating the CPN Type model, EMF can generate source code for interfaces
and implementations of the new classes. This is done by creating a ``genfile''
linked to the EMF model. The genfile can define meta-info such as the base
package of generated source files, and configuration parameters for the
individual classes. EMF can then generate different groups of code, but for an
ePNK Type we only need Model code and Edit code. the Model code includes
interfaces and corresponding basic implementations of the classes as well as
factories for instantiating them, while the Edit code contains classes for
presenting and manipulating the model in an editor.

After generating the code for the Type Model, the source file for
the implementation of the PetriNetType subclass needs two minor modifications
to work with ePNK: The constructor must be made public (it is protected by
default), and the toString method must be implemented to conform to the
PetriNetType interface. This method should return a string that textually
represents the net type, usually simply its formal name.

Before ePNK will recognise the plugin and the Type Model, the plugin manifest
needs to be edited to define this plugin as an extension to the
org.pnml.tools.epnk.pntd extension point of ePNK. All that is needed to
configure this is supplying a unique id, a descriptive name, and the fully
qualified classpath to the PetriNetType subclass. The resulting
element in plugin.xml is shown in \lstref{pntd_xml}. 

\begin{lstlisting}[language=XML,label=lst:pntd_xml,float,
caption=ePNK Petri Net Type Definition Extension]
   <extension
         id="org.cpntools.pragma.epnk.pnktypes.cpndefinition"
         name="CPN"
         point="org.pnml.tools.epnk.pntd">
      <type
            class="org.cpntools.pragma.epnk.pnktypes.cpndefinition.impl.CPNImpl"
            description="Coloured Petri Net from CPN Tools">
      </type>
   </extension> 
\end{lstlisting}

\subsection{Constraints}

We define one constraint for the CPN Type: An Arc must go between a Place and a
Transition. Technically this is done via the EMF extension point
``org.eclipse.emf.validation.constraintProviders''. There are several ways of
defining extensions for this point, some of which are explained in
\cite{kindler2011epnk}. Our extension declaration is shown in
\lstref{constraintextension}, and is configured to use a Java class to perform
validation (defined in the \textless{}constraint\textgreater{} element's class
attribute).

\begin{lstlisting}[language=XML,label=lst:constraintextension,float,numbers=left,stepnumber=1,
caption=constraintProvider Extension] 
   <extension
         point="org.eclipse.emf.validation.constraintProviders">
      <category
            id="org.pnml.tools.epnk.validation"
            name="CPN Validation">
      </category>
      <constraintProvider
            cache="true"
            mode="Batch">
         <package
               namespaceUri="http://org.pnml.tools/epnk/pnmlcoremodel">
         </package>
         <constraints
               categories="org.pnml.tools.epnk.validation">
            <constraint
                  class="org.cpntools.pragma.epnk.pnktypes.cpndefinition.validation.ArcSourceAndTargetLimitation"
                  id="org.cpntools.pragma.epnk.pnktypes.cpndefinition.validation.ArcSourceAndTargetLimitation"
                  isEnabledByDefault="true"
                  lang="Java"
                  mode="Batch"
                  name="Arc source and target limitation"
                  severity="ERROR"
                  statusCode="301">
               <target
                     class="Arc:http://org.cpntools/pragma/epnk/pnktypes/cpndefinition">
               </target>
               <description>
                  An Arc must go between a Place and either a Transition or a Page.
               </description>
               <message>
                  The source and target of arc {0} are not compatible.
               </message>
            </constraint>
         </constraints>
      </constraintProvider>
   </extension>
\end{lstlisting}

The class ArcSourceAndTargetLimitation we have specified is shown in
\lstref{constraintimpl}.
The IValidationContext argument provides both the target model element to be
validated, and facilities for creating an appropriate return object representing
failure or success. The algorithm for checking the arc's source and target is
straightforward.

\begin{lstlisting}[language=Java,label=lst:constraintimpl,float,numbers=left,stepnumber=1,
caption=Constraint Implementation]
public class ArcSourceAndTargetLimitation extends AbstractModelConstraint {
	public IStatus validate(IValidationContext ctx) {
		EObject eObj = ctx.getTarget();
		if (eObj instanceof Arc) {
			Arc arc = (Arc) eObj;
			Node source = arc.getSource();
			Node target = arc.getTarget();
			
			if (source != null && target != null) {
				PlaceNode pn = null;
				Node other = null;
				if(source instanceof PlaceNode) {
					pn = (PlaceNode) source;
					other = target;
				} else if (target instanceof PlaceNode) {
					pn = (PlaceNode) target;
					other = source;
				}
				if(	pn == null || // there was no Place
					other instanceof PlaceNode) // both are Places
					return ctx.createFailureStatus(new Object[] {arc});
			} 
		} 
		return ctx.createSuccessStatus();
	}
}
\end{lstlisting}

\subsection{Annotated CPN Type}
The Annotated CPN Type model is shown in \figref{ann_model_diagram}. It extends
the CPN model to enable annotation of model elements. It also handles references
to ontology documents that define sets of pragmatics.

\fig[0.45]{PragmaCPNDefinition.pdf}{Annotated CPN model type diagram}
{ann_model_diagram}

As with the CPN Type Model, we have subclassed the classes of model elements we
need to extend. All of these need to be extended to support the same feature:
being annotated with pragmatics. Ideally, we would have changed a superclass of
these classes, for instance Node (see \figref{pnmlcoremodel}). But this will not
work, since model merging is not possible (as discussed earlier), and
subclassing Node will not work, since existing classes will not be subclasses of
our new superclass and therefore not inherit anything from it.

We needed to devise a way to extend a group of classes without access to their
superclass, and ideally without copying code between the classes. The solution
was to define an interface that all of the target classes
implement\footnote{Note that in \figref{ann_model_diagram} the arcs that signify
this are easily mistaken as multiple inheritance. Only the small (I) icon on
the OntologyMember element shows that this is an interface, and not a class.}.
This is the OntologyMember interface. OntologyMember has a reference to the Pragma class.
The reference is configured to act as containment, meaning Pragma instances are
created as children of OntologyMember instances. 

A Pragma instance represents a pragmatic annotation, and stores the IRI of the
ontology class to which it belongs. Pragmatics can also define configuration
arguments. Like the inscriptions from the CPN Type, Pragma inherits the Label
class to be able to be displayed in the diagram editor. The text attribute is
configured to be virtual\footnote{This means no field is generated and nothing
is serialised, and is often used for attributes that are purely derived from
other attributes.}, and manually implemented to return the pragmatics class
enclosed in \textless\textless and \textgreater\textgreater, including arguments
if present.

\subsection{Persisting Pragmatics Sets in the Model}
We have already shown how pragmatics sets can be defined using ontology
documents. But we also need a way to include them in the models we want to
annotate. In other words: a model needs to store which pragmatic sets it
utilises. This is encapsulated in the OntologyDocument class
(\figref{ann_model_diagram}, lower right), which can store the IRI of the
ontology and the path to the document that contains it. This is sufficient to be
able to reload the ontology if the CPN model is closed and later reopened.

These OntologyDocument instances need to be serialised somewhere in the model.
Our first idea was to model them as contained in the PragmaCPN class (i.e.
the Type), however due to the fact that this element is serialised only as an
attribute this was not possible. We decided to create a new class which instance
is contained by the PetriNet class. This is the PragmaticsOntology class
(\figref{ann_model_diagram}, right), and the intention is to have a central
entity to manage the ontologies of the entire net, including loading and
reasoning. For these purposes we defined one property to contain the OWL API
ontology manager, as well as two operations for including new ontology documents
and deducing which pragmatics are valid for a given OntologyMemeber. These will
be described later in the chapter.

\subsection{Adding Pragmatics Sets}\label{sec:menufirsttime}
There needs to be a convenient way to add domain and model specific pragmatics
sets to a model from ontology documents. We decided to extend the context menu
of the PragmaticsOntology class for this purpose.

Eclipse defines several extension points for extending every part of the UI.
Menus in particular have more than one way of being extended. The ePNK manual
suggests using the |org.eclipse.ui.popupMenus| extension point and has an
example of how to implement such an extension. By following this example we have
created the menu command ``Add Ontology'' under the submenu Pragmatics. The
extension is listed in \lstref{addOntologyMenu}. 

\begin{lstlisting}[language=XML,float,label=lst:addOntologyMenu,numbers=left,stepnumber=1,
caption=Add Ontology Menu Extension]
   <extension
         point="org.eclipse.ui.popupMenus">
      <objectContribution
            adaptable="false"
            id="ePNK Annotated CPN Type.edit.objectContribution1"
            objectClass="org.cpntools.pragma.epnk.pnktypes.pragmacpndefinition.PragmaticsOntology">
         <menu
               id="org.cpntools.pragma.epnk.pnktypes.pragmacpndefinition.actions.standardmenu"
               label="Pragmatics"
               path="additions">
            <separator
                  name="group1">
            </separator>
         </menu>
         <action
               class="org.cpntools.pragma.epnk.pnktypes.pragmacpndefinition.menu.AddOntologyAction"
               enablesFor="1"
               id="org.cpntools.pragma.epnk.pnktypes.pragmacpndefinition.actions.AddOntologyAction"
               label="Add Ontology"
               menubarPath="org.cpntools.pragma.epnk.pnktypes.pragmacpndefinition.actions.standardmenu/group1">
         </action>
      </objectContribution>
   </extension>
\end{lstlisting}

Every menu contribution should supply an |id| attribute to allow it to be
referenced by external code. This is the case for all the elements in this
listing, so we explain it here to avoid repetition.

The extension defines an
|<objectContribution>| element (line 3-6), stating that this menu item should
only appear in the context menu when an object of the specified type is selected. The
|adaptable| attribute is only relevant to IResource objects, but is a required
attribute with a default value of false. The |objectClass| is the
target class that will have its context menu extended with the new menu, in this
case our PragmaticsOntology class.

We then define a |<menu>| element (line 7-14) that will create a submenu
intended to group together all menu contributions related to pragmatics. The
|label| is simply the label, and |path| specifies where in the parent menu it
should be placed, with the value ``additions'' being recommended. The menu has
a |separator| child element (line 11-13) that acts as an anchor for positioning
items.

Last, there is the |action| element (starting line 15), defining the menu item
that performs the action. It specifies the |class| that performs an action, which should be a
class implementing the IObjectActionDelegate interface. We have implemented this
as the |AddOntologyAction| class, described below. The |enablesFor| attribute
specifies that the action is enabled (clickable) when exactly one element is
selected. |label| is the displayed label. |menubarPath| defines
placement by referencing the |id| and |separator| we defined for |<menu>|.

As soon as a PragmaticsOntology element is right-clicked and the menu is about
to be displayed, an |AddOntologyAction| instance will be created and configured
with |setActivePart()| (which allows us to get the current shell to be able to display error messages)
and |selectionChanged()| (which gives us the current model selection). These two
methods are shown in \lstref{AddOntologyAction_config}.


\begin{lstlisting}[language=Java,float,label=lst:AddOntologyAction_config,
caption=AddOntologyAction setActivePart() and selectionChanged()]
	private PragmaticsOntology ontology;
	private Shell shell;

	@Override
	public void selectionChanged(IAction action, ISelection selection) {
		ontology = null;
		if (selection instanceof IStructuredSelection) {
			IStructuredSelection structuredSelection = (IStructuredSelection) selection;
			if (structuredSelection.size() == 1
					&& structuredSelection.getFirstElement() instanceof PragmaticsOntology) {
				ontology = (PragmaticsOntology) structuredSelection.getFirstElement();
			}
		}
		action.setEnabled(ontology != null);
	}

	@Override
	public void setActivePart(IAction action, IWorkbenchPart targetPart) {
		shell = targetPart.getSite().getWorkbenchWindow().getShell();
	}
\end{lstlisting}

When the menu item is clicked, the |run()| method will be called, shown in
\lstref{AddOntologyAction_run}. This method uses standard Eclipse classes to
open a file selection dialog. This dialog shows the current project's resources
and asks the user to select an ontology document. It does not filter files on
their extension, as there exist editors that save ontologies with other
extensions than those defined by W3C.

\begin{lstlisting}[language=Java,float,label=lst:AddOntologyAction_run,
caption=AddOntologyAction run()]
	@Override
	public void run(IAction action) {
		ElementTreeSelectionDialog dialog = new ElementTreeSelectionDialog(
				shell, new WorkbenchLabelProvider(),
				new BaseWorkbenchContentProvider());
		dialog.setTitle("Select an Ontology");
		dialog.setMessage("Select the Ontology document you want to include in the net:");
		
		dialog.setInput(getProject(ontology));
		dialog.open();

		if (dialog.getReturnCode() == Window.OK) {
			Object o = dialog.getFirstResult();
			if (o instanceof IFile) {
				IFile f = (IFile) o;
				ontology.addOntologyFromFile(f);
			} else {
				IStatus status = new Status(IStatus.ERROR, "PragmaCPN", 0,
						"Invalid file", null);
				ErrorDialog.openError(shell, "Invalid file",
						"The selected file was not valid.", status);
			}
		}
	}
\end{lstlisting}

If the user selects a file and clicks OK, the |run()| method continues by
calling |addOntologyFromFile()| on the selected |PragmaticsOntology|. This
method is shown in \lstref{addOntologyFromFile}. All modification of an EMF
model should happen through an EditingDomain. This is a facility that integrates
with Eclipse to provide undo/redo capabilities, as well as ensuring 
model changes are performed in an atomic manner. We use it by creating an
appropriate Command and executing it on the EditingDomain's CommandStack. In
this case we create an AddCommand to add an OntologyDocument as a child of the
PragmaticsOntology. 

\begin{lstlisting}[language=Java,float,label=lst:addOntologyFromFile,
caption=addOntologyFromFile()]
	public void addOntologyFromFile(IFile file) {
		try {
			TransactionalEditingDomain domain = TransactionUtil.getEditingDomain(this);

			if (domain != null) {
				OWLOntology ont = loadOntology(file);
				//System.out.println(ont);
				if(ont.getOntologyID().getOntologyIRI() == null){
					MessageDialog.openError(getShell(), "Load Error", 
							"The file does not seem to contain a valid ontology.");
					unloadOntology(ont);
					return;
				}
				OntologyDocument loadedDoc = PragmacpndefinitionFactory.eINSTANCE.createOntologyDocument();
				loadedDoc.setIri(ont.getOntologyID().getOntologyIRI().toString());
				loadedDoc.setPath(file.getFullPath().toPortableString());
				//System.out.println(getManager().getOntologies().size());
				domain.getCommandStack().execute(
					AddCommand.create( domain, this, 
						PragmacpndefinitionPackage.PRAGMATICS_ONTOLOGY__DOCUMENTS, loadedDoc
				));
			}
		} catch (OWLOntologyCreationException e) {
			e.printStackTrace();
			String s = "";
			String msg = e.getMessage();
			if(msg != null && msg.length()>0)
				s = "Reason: " + msg;
			MessageDialog.openError(getShell(), "Load Error", 
					"The file could not be loaded. "+s);
		} catch (CoreException e) {
			e.printStackTrace();
		}
	}
\end{lstlisting}


\section{Importing from CPN Tools}\label{sec:import_wizard}
Access/CPN is a framework that can parse CPN models saved by CPN Tools and
represent the model with EMF classes. Access/CPN has many additional features
related to the semantics of CPN, but only the model importer is of relevance for
the work in this thesis.

Access/CPN also uses EMF to represent models internally. The EMF model for CPN
models that Access/CPN defines uses many of the same class names as ePNK, which
makes it tedious to write and read code working between the two frameworks due
to the need to use fully qualified classpaths to avoid class name collisions.
Initially we planned to extract the parser source code from Access/CPN and
rewriting it to use the new ePNK CPN Type classes. This plan was later discarded
in favor of depending on the Access/CPN plugin, as Access/CPN has continually
been improved during development of \thename{}, and is now also capable of
parsing graphics data. And by depending on Access/CPN instead of writing our own
parser, we can benefit from further updates and improvements.

The Eclipse Plugin IDE includes several template plugin projects, and one of
them is for creating an Import Wizard, that is a step-by-step dialog for
importing resources into the Eclipse workbench, for instance copying a file from
the file system to a project, or importing entire projects from version control.
The template also pre-configures the plugin manifest with an extension that
makes the wizard available in Eclipse's Import Dialog. This is shown in
\figref{import_wizard_1} (a)

The import wizard for importing CPN Tools models only has one step, which
requires the user to select the .cpn file from the file system, and select a
destination in the workspace for the converted model. It is shown in
\figref{import_wizard_1} (b). After selecting the source file, the wizard
suggests a new name for the converted model based on the original file name but
with the .pnml extension used by ePNK.

\figDbl[0.4]{import_wizard_1.png}{Import Dialog}{import_wizard_2.png}{CPN Tools
Importer}{The Import Wizard}{import_wizard_1}

The conversion process is straightforward, the .cpn file is loaded with
Access/CPN, and the resulting model is then converted object by object to the
ePNK Annotated CPN type. The final model instance also includes a
PragmaticsOntology instance.

\section{Creating Pragmatic Annotations}
We want it to be fast and easy to annotate model elements, and an immediate
example to follow is the way child and sibling elements are added to the model
through context menus. It is possible to create an empty Pragma instance as a
child of an OntologyMember instance without any manual programming; ePNK and EMF
dynamically takes care of that. The user can then manually enter the IRI of the
desired pragmatic.

However, we have also implemented a new menu entry similar to New Child and New
Sibling that is dynamically populated with all valid pragmatics for the selected
model element. The routine for providing the menu entry can be divided into
two parts, and we will explain each part in the following subsections.

	\subsection{Providing a Dynamic Context Menu}
	In Section \ref{sec:menufirsttime} we described a simple way of extending popup
	menus with static entries through the org.eclipse.ui.popupMenus extension
	point. This point can however not be used to provide a menu with dynamically
	generated menu items, which we need for domain- and model-specific pragmatics.
	Instead, we need to use the org.eclipse.ui.menus extension point, which allows
	more generic menu extensions. 
	
	The extension we created is shown in
	\lstref{addPragmaMenu}. The locationURI attribute is this time set to the URI
	of the editor itself. The menu is defined by a |<dynamic>| element pointing
	to the class AnnotationContributionMenu (described shortly), and a
	|<visibleWhen>| element constraining this menu to only be visible when we have
	selected exactly one OntologyMember instance in the model.
	
	\begin{lstlisting}[language=XML,float,label=lst:addPragmaMenu,numbers=left,stepnumber=1,
	caption=Add OntologyPragmatic Dynamic Menu Extension]
   <extension
         point="org.eclipse.ui.menus">
      <menuContribution
            locationURI="popup:org.pnml.tools.epnk.pnmlcoremodel.presentation.PnmlcoremodelEditorPNML?before=additions">
         <menu
               label="Annotate with">
            <dynamic
                  class="org.cpntools.pragma.epnk.pnktypes.pragmacpndefinition.menu.AnnotationContributionMenu"
                  id="org.cpntools.pragma.epnk.pnktypes.pragmacpndefinition.menu.AnnotationContributionMenu">
            </dynamic>
            <visibleWhen>
               <and>
                  <count
                        value="1">
                  </count>
                  <iterate>
                     <instanceof
                           value="org.cpntools.pragma.epnk.pnktypes.pragmacpndefinition.OntologyMember">
                     </instanceof>
                  </iterate>
               </and>
            </visibleWhen>
         </menu>
      </menuContribution>
   </extension>
	\end{lstlisting}
	
	The first part of the AnnotationContributionMenu class is shown in
	\lstref{AnnotationContributionMenu}. To be able to be used as a dynamic menu,
	it should implement the IContributionItem interface, and there exist several
	classes that implement this interface that can be further subclassed for the
	desired purpose. In this case, we subclass the CompoundContributionItem class
	which is designed to provide a dynamic list of items by calling 
	|getContributionItems()| when the menu is about to be shown.
	
	
	\begin{lstlisting}[language=Java,float,label=lst:AnnotationContributionMenu,numbers=left,stepnumber=1,
	caption=AddOntologyAction run()]
	public class AnnotationContributionMenu extends CompoundContributionItem {
	
	private TransactionalEditingDomain domain;
	
	protected IContributionItem[] getContributionItems() {
		OntologyMember member = getSelectedItem();
		domain = TransactionUtil.getEditingDomain(member);
		PragmaticsOntology ontology = resolveOntology(member);
		if(ontology == null) 
			return createDisabledMenu("Missing PragmaticsOntology");
		
		Set<OWLClass> pragmas;
		try {
			pragmas = ontology.getValidPragmatics(member);
		} catch (CausesInconcistencyException e) {
			IContributionItem[] item = {new ActionContributionItem(
					new ShowInconsistencyErrorAction(e.getSet(), getWorkbench().getShell())
					)};
			return item;
		}
		if(!pragmas.isEmpty()) {
			List<IContributionItem> items = new ArrayList<IContributionItem>();
			for(OWLClass pragma : pragmas) {
				items.add(new ActionContributionItem(
						AddPragmaAction.create(domain, getSelection(), 
							pragma.getIRI().toString())));
			}
			return items.toArray(new IContributionItem[items.size()]);
		}
		return createDisabledMenu("No applicable pragmatics");
	}
	\end{lstlisting}
	
	Aside from some utility methods (which we have omitted in the listing), we call
	getValidPragmatics (described in the next section) on the model's
	PragmaticsOntology instance (if one exists).
	If we get a populated set back, we generate an ActionContributionItem for
	each class in the set. An ActionContributionItem is a simple type of
	contribution item that will execute its assigned IAction when clicked. Each of
	the items are given an AddPragmaAction instance, a private class designed to
	provide a more convenient interface for creating 
	CreateChildAction instances. The AddPragmaAction class is shown in
	\lstref{AddPragmaAction}. This type of action wraps the low-level API of the
	CommandStack and related classes to yield a more concise way to
	encapsulate changes to a model, in this case adding a Pragma child to the
	selected element. The current selection should also be provided in case it
	needs to be restored, for example if the user chooses to undo it.
	
	\begin{lstlisting}[language=Java,float,label=lst:AddPragmaAction,
	caption=AddPragmaAction]
	private static class AddPragmaAction extends CreateChildAction {
		private AddPragmaAction(EditingDomain editingDomain,
				ISelection selection, Object descriptor) {
			super(editingDomain, selection, descriptor);
		}

		public static AddPragmaAction create(EditingDomain editingDomain,
				IStructuredSelection selection, String pragmaIRI) {
			Pragma p = PragmacpndefinitionFactory.eINSTANCE.createPragma();
			p.setIri(pragmaIRI);
			CommandParameter desc = new CommandParameter(selection.getFirstElement(), PragmacpndefinitionPackage.Literals.ONTOLOGY_MEMBER__ANNOTATION, p);
			AddPragmaAction action = new AddPragmaAction(editingDomain, selection, desc);
			action.setText(pragmaIRI);
			return action;
		}
	}
	\end{lstlisting}
	
	\subsection{Determining Appropriate Pragmatics}\label{sec:cwa}
	
	As mentioned above, the PragmaticsOntology is responsible for determining
	appropriate pragmatics for the current model selection through the
	getValidPragmatics() method. We will describe it in parts throughout this
	section.
	
	In broad terms, the algorithm we designed is na\"{\i}ve in its nature: Every
	time the method is called (which happens when the Add Pragmatics submenu is
	expanded), it will use OWL API to create a new ontology from scratch to
	describe the current state of the model. It then adds one new pragmatic
	instance to the ontology, and for each subclass defined for Pragmatic, it
	checks if the ontology is still consistent if the pragmatic is of that class,
	and if so, this pragmatic class is valid.
	
	\lstref{validPragma1} begins with some static values which will be used
	throughout the method. The PrefixManagers purpose is equivalent to the Prefix()
	statements used in written ontologies. Continuing, we have the beginning of the
	getValidPragmatics() method, which takes as its argument the target model
	entity to receive a pragmatic. This must be an OntologyMember, so the variable
	is named |member|.
	
	We call createModelOntology(), which creates an empty ontology, imports the
	CPN, Basic and General Pragmatics ontologies as well as any ontologies loaded
	in the model, and for each model element declares an individual with
	appropriate class and properties. The name used to declare the individual is
	the entity's id attribute, which must be present for the model to be valid.
	Existing pragmatics are also asserted as individuals, using the object hash as
	an id.
	
	\begin{lstlisting}[language=Java,float,label=lst:validPragma1,
	caption=getValidPragmatics Part 1]
		static final private String cpnurl = "http://hib.no/ontologypetrinets/cpn/";
		static final private String basicurl = "http://k1s.org/OntologyRestrictedNets/basic/";
		static final private String nppnurl = "http://org.k1s/orn/nppn/";
		static private String localurl = "http://local.model/";
		static private PrefixManager cpnPrefix = new DefaultPrefixManager(cpnurl);
		static private PrefixManager basicPrefix = new DefaultPrefixManager(basicurl);
		static private PrefixManager nppnPrefix = new DefaultPrefixManager(nppnurl);
		static private PrefixManager localPrefix = new DefaultPrefixManager(localurl);
	
		public Set<OWLClass> getValidPragmatics(OntologyMember member) throws CausesInconcistencyException {
		// Create an empty ontology
		OWLOntology ontology;
		try {
			ontology = createModelOntology();
		} catch (OWLOntologyCreationException e) {
			e.printStackTrace();
			return Collections.emptySet();
		}
	\end{lstlisting}
	
	\lstref{validPragma2} shows the second part of the getValidPragmatics method.
	Interaction with the OWL API happens by using an OWLDataFactory to create
	objects representing different concepts. The factory is acquired from an OntologyManager, which is
	acquired from the Ontologies plugin (discussed in Section
	\ref{sec:ontologiesPlugin}). We also create a ReasonerFactory, in this case
	for a PelletReasoner. 
	
	Next, we create an object representing the pragmatic we are about to add to the
	model, as well as one for the target member. We then assert the hasPragmatic
	axiom from the member to the pragmatic. Last, we use the manager to add this
	axiom to our ontology.
	
	\begin{lstlisting}[language=Java,float,label=lst:validPragma2,
	caption=getValidPragmatics Part 2]
		// Get factories
		OWLDataFactory dataFactory = getManager().getOWLDataFactory();
		PelletReasonerFactory reasonerFactory = PelletReasonerFactory.getInstance();
		
		// Create mock pragmatic and get target member
        OWLNamedIndividual pragmaInd = dataFactory.getOWLNamedIndividual(":_pragma",localPrefix);
        OWLNamedIndividual memberInd = dataFactory.getOWLNamedIndividual(":"+member.getId(), localPrefix);
        // Create axiom for "member hasPragmatic _pragma"
		OWLObjectProperty hasPragmatic = dataFactory.getOWLObjectProperty(":hasPragmatic", basicPrefix);
		OWLObjectPropertyAssertionAxiom hasPragmaticAxiom = 
				dataFactory.getOWLObjectPropertyAssertionAxiom(hasPragmatic, memberInd, pragmaInd);
		getManager().applyChange(new AddAxiom(ontology, hasPragmaticAxiom));
	\end{lstlisting}
	
	\lstref{validPragma3} shows the next steps, consisting of creating axioms for
	enforcing Closed World Assumption (CWA). As discussed in the last chapter, OWL
	2 uses an Open World Assumption, which states that any fact can be true unless
	explicitly asserted to be false. This implies there may exist other
	individuals and property instances other than the ones we have asserted to
	exist. This will prevent us from properly evaluating complex pragmatics like
	:External that define their availability based on the existence of other
	pragmatics. 
	
	In order to reason under CWA, three measures must be taken:
	\begin{itemize}
		\item Classes that are disjoint must be asserted as such. Otherwise having an
		individual be both a Place and an Arc at the same time would not entail an
		inconsistency. This needs to be designed manually, and is already included in
		the ontologies we have described.
		\item The universe of individuals must be limited to only include the
		individuals we have declared. The class owl:Thing represents
		every individual, asserted or otherwise. We must assert it to be equivalent to
		the set of the individuals we have explicitly declared (which is trivial to
		retrieve from the ontology).
		\item It is necessary to assert every property that is not explicitly stated
		to not exist. We do this for the :hasPragmatic property by first asserting all
		individuals to be different, and then for each individual set the cardinality
		of the :hasPragmatic property to be exactly the number of already declared
		values for that individual on that property. Combined, these assertions
		effectively limit the existence of properties to those we explicitly declared.
	\end{itemize}
	
	\begin{lstlisting}[language=Java,float,label=lst:validPragma3,
	caption=getValidPragmatics Part 3]
		// Axioms for closed world assumption: 
		//   Limiting individuals to those declared
		Set<OWLNamedIndividual> allInds = ontology.getIndividualsInSignature(); System.out.println(allInds.size());
		OWLObjectOneOf oneOfAllInds = dataFactory.getOWLObjectOneOf(allInds);
		OWLEquivalentClassesAxiom cwaAxiom = dataFactory.getOWLEquivalentClassesAxiom(dataFactory.getOWLThing(), oneOfAllInds);
		getManager().applyChange(new AddAxiom(ontology, cwaAxiom));
		//   Ensuring they are different individuals
		OWLDifferentIndividualsAxiom diffInds = dataFactory.getOWLDifferentIndividualsAxiom(allInds);
		getManager().applyChange(new AddAxiom(ontology, diffInds));
		//   Asserting object property cardinality, effectively
		//   limiting property values to those explicitly declared 
		Set<OWLClassAssertionAxiom> cardAxioms = new HashSet<OWLClassAssertionAxiom>();
		for(OWLNamedIndividual ind : allInds){
			Set<OWLIndividual> pragmas = ind.getObjectPropertyValues(hasPragmatic, ontology);
			OWLObjectExactCardinality cardExp = dataFactory.getOWLObjectExactCardinality(pragmas.size(), hasPragmatic);
			OWLClassAssertionAxiom cardAxiom = dataFactory.getOWLClassAssertionAxiom(cardExp, ind);
			cardAxioms.add(cardAxiom);
			getManager().applyChange(new AddAxiom(ontology, cardAxiom));
		}
	\end{lstlisting}
	
	The last part of getValidPragmatics is shown in
	\lstref{validPragma4}. We now create a reasoner instance for the ontology that
	has been constructed. We first use it to check if the ontology is consistent,
	which is done by determining if the ontology entails that owl:Thing is
	equivalent to owl:Nothing, in essence meaning there exists a contradiction.
	This is done to ensure that the new pragmatic does not violate conditions for
	any existing pragmatics.
	
	Next, we determine all subclasses of basic:Pragmatic, and iterate over them.
	For each one, we create an axiom asserting the new pragmatic belongs to this
	subclass, check if the ontology is still consistent, and if so then include it
	in the list of valid pragmatics. We then remove the subclass axiom to prepare
	for the next subclass.
	
	Finally, when all subclasses have been checked, we dispose of the reasoner (to
	save memory) and return the subclasses we have determined to be valid.
	
	\begin{lstlisting}[language=Java,float,label=lst:validPragma4,
	caption=getValidPragmatics Part 4]
		// Create a reasoner and an explainer
		OWLReasonerConfiguration config = new SimpleConfiguration(new ConsoleProgressMonitor());
		PelletReasoner reasoner = reasonerFactory.createNonBufferingReasoner(ontology, config);
		
		// Check for inconsistency
		if(!reasoner.isConsistent()) {
			reasoner.dispose();
			// Throw exception
			throw new CausesInconcistencyException();
		}
		
		// Get all Pragmatic subclasses
		Set<OWLClass> classes = new HashSet<OWLClass>();
        OWLClass pragmaClass = dataFactory.getOWLClass(":Pragmatic", basicPrefix);
		NodeSet<OWLClass> pragmaSubClasses = reasoner.getSubClasses(pragmaClass, false);
		
		// Iterate the subclasses
		for(OWLClass pragmaclass : pragmaSubClasses.getFlattened()) {
			// Declare the mock pragma individual as subclass of current iteration item
			OWLClassAssertionAxiom newClass = dataFactory.getOWLClassAssertionAxiom(pragmaclass, pragmaInd);
			getManager().applyChange(new AddAxiom(ontology, newClass));
			// Check if ontology is still consistent, and include class if it is
			if(reasoner.isConsistent()) {
				classes.add(pragmaclass);
			}
			// Remove the subclass declaration axiom
			getManager().applyChange(new RemoveAxiom(ontology, newClass));
		}
		
		reasoner.dispose();
		return classes;
	}
	\end{lstlisting}
	
	
	
\section{Defining Model Specific Pragmatics Sets}
Creating a set of model specific pragmatics (or any set of pragmatics) is as
simple as writing an ontology for it, but this is not easy if one is not
already familiar with ontologies and the OWL 2 specification and syntaxes. There
exist a number of tools for writing ontologies, both textual and graphical. We
currently use a basic textual editor for Eclipse with rudimentary syntax
highlighting and content assistance tailored for pragmatics, which will be
included in the release of \thename{}.
We examine some ideas on how the editor can be improved in Chapter
\ref{chap:conclusion}.
	
\section{Assessment of Requirements}
We have successfully met all four requirements that were outlined in Section
\ref{sec:requirements}.

\begin{itemize} 
	\item By using Access/CPN, we have made \thename{} capable of importing
	models created with CPN Tools.
	\item Models can be annotated using our extension of the ePNK framework.
	Pragmatics are provided using the strong expressiveness of ontologies, and by
	using an ontology reasoner we show only relevant pragmatics based on the
	selected model element.
	\item The editor only uses ontologies specified in the target model, letting
	the user choose which domain specific pragmatics to have available.
	\item The user is able to define model specific pragmatics by writing an
	ontology to describe them, which can then be included in the model the same way
	as domain specific pragmatics.
\end{itemize}
