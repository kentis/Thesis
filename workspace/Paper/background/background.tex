\chapter{The WebSocket Protocol CPN Model}
\label{chap:background}

\section{The WebSocket Protocol}
The primary case study for this thesis is the WebSocket protocol 
\cite{draft-ietf-hybi-thewebsocketprotocol}. From the abstract of the document:
\begin{quote}The WebSocket protocol enables two-way communication between a
client running untrusted code running in a controlled environment to a
remote host that has opted-in to communications from that code.
\end{quote}

\fig[0.4]{WebSocketSequenceDiagram.pdf}{Sequence Diagram of the WebSocket
protocol}{wssequence}

\figref{wssequence} shows the basic sequence of the WebSocket protocol. To
establish a connection, a client sends a specially formatted HTTP request to a
server, which replies with a HTTP response. Once the connection is
established, the client and server can then freely send WebSocket message
frames, until either endpoint sends a control frame with the opcode 0x8 for
close and optionally data about the reason for closing. The other endpoint then
replies with the same opcode and data, and the connection is considered closed.

From the RFC document:

\begin{quote}
	Conceptually, WebSocket is really just a layer on top of TCP that
	does the following:
	\begin{itemize}
		\item adds a Web "origin"-based security model for browsers
	
		\item adds an addressing and protocol naming mechanism to support
		multiple services on one port and multiple host names on one IP
		address;
	
		\item layers a framing mechanism on top of TCP to get back to the IP
		packet mechanism that TCP is built on, but without length limits
	
		\item includes an additional closing handshake in-band that is designed
		to work in the presence of proxies and other intermediaries
	\end{itemize}
\end{quote}

\section{Coloured Petri Nets}

Coloured Petri Nets (CPN) are a type of directed graph used to model processes,
especially processes with an asynchronous and/or concurrent nature. Common
examples are modelling networks, processes and protocols, as well as
concurrent programming design.

The strength of CPN lies i the operations that can be performed with it:
Simulation, verification and analysis. 

More recently, research has been conducted to examine the feasibility of using
CPN for software design with automatic code generation.

The structure of CPN models will be explained gradually through the case study
in this chapter.

\subsection{CPN Tools}

CPN Tools \cite{cpntools} is a popular graphical
tool for working with CPN models, from construction and simulation, to analysis
via state space exploration.

	CPN Tools uses the CPN ML language to specify declarations and net inscriptions.
	This language is an extension of the functional programming language Standard
	ML \cite{milner1997definition}, developed at Edinburgh University.
	
\section{The WebSocket CPN Model}

The websocket protocol is the primary case study of this thesis. 
In this section the model that has been built is described in full detail. 

A CPN Model is organised into pages, also called modules. Each module can also
contain sub-modules, which can help keep complex and/or large models clear and
manageable. The WebSocket model makes full use of this feature, and will be
described in a top-down manner.

\subsection{Overview}

	\fig{Overview.eps}{Overview of CPN model of the WebSocket protocol}{overview}
	
	\figref{overview} shows the top-level Overview module of the
	WebSocket CPN model. It consists of five sub-modules represented by double
	border rectanges, and several places and arcs that connect them, represented
	with circles and arrows respectively. The function of arcs will be explained
	under the next submodule.
	
	The sub-modules have been laid out to resemble part of the OSI model
	\cite{osi7}, where Client Application and Server Application each correspond to
	the top two layers Application and Presentation, Client WebSocket and Server
	WebSocket correspond to the Session layer, and Connection corresponds to the
	lower layers Transport, Network, Data Link and Physical.
	
	Places are used to represent the state of the modelled system, and can contain
	tokens. Each token can have a data value, termed the token colour. The number
	of tokens and their colors in a specific place is termed the marking of that
	place. Similarly, the tokens in all places in the model together form the 
	marking of the model, and thus represents the state of the system.
	
	%\subsubsection{Declarations and Inscriptions}
	
	All colour sets, variables, symbolic constants and functions have to be
	declared globally for the model. Colour sets are defined with the following syntax:
	
	|colset <name> = <type-specification>;|
	
	A colour set is defined as a range or set of data
	values, and is declared globally for the model using CPN ML. Colour set names
	are always capitalised in this thesis, but any CPN ML identifier is valid. A
	simple example is |colset INT = int;|, representing the set of all integers. A
	token from this colour set could for example have the value '3'. Similar
	declarations are present for |STRING|, |BOOL| and |UNIT|.
	These are examples of simple color sets. 
	
	It is also possible to delcare more complex color sets, termed compound color
	sets, to form data structures of data types. One such compound structure is the
	list, used to define an ordered collection of tokens from one color set.
	
	Colour sets defined in the WebSocket model will be explained as they are
	encountered in each sub-module. The places in the Overview use the following
	colour set definitions:
	
	\begin{lstlisting}[label=lst:overview_colset,caption=Overview colour sets,gobble=1] 
	colset OPERATION = with TEXT | BINARY | PING | PONG | CLOSE; 
	colset MESSAGE = record Op: OPERATION *	Message: STRING;
	colset MESSAGES = list MESSAGE;
	
	colset URL = record Protocol: STRING * Host: STRING * Port: INT * Path: STRING;
	colset CLIENT_CALL = union Connect:URL + CliSendMsg:MESSAGE;
	colset CLIENT_CALLS = list CLIENT_CALL;
	
	colset CONN_RESULT = bool with (fail, success);
	colset CLIENT_EVENT = union CliGetMsg:MESSAGE + ConnResult:CONN_RESULT;
	colset CLIENT_EVENTS = list CLIENT_EVENT;
	
	colset CONN_REPLY = bool with (reject, accept);
	colset SERVER_CALL = union SerSendMsg:MESSAGE + ConnReply:CONN_REPLY;
	colset SERVER_CALLS = list SERVER_CALL;
	
	colset SERVER_EVENT = union SerGetMsg:MESSAGE + ConnRequest:UNIT;
	colset SERVER_EVENTS = list SERVER_EVENT;
	
	colset PACKET = union HttpReq:HTTPREQ + HttpRes:HTTPRES + WsFrame:WSFRAME;
	colset PACKETLIST = list PACKET;
	\end{lstlisting}
	
	The |OPERATION| colour set is an enumeration that represents the
	different types of messages that can be passed between the application and the
	protocol layer. All of these correspond with opcodes used in WebSocket frames.
	
	The |MESSAGE| colour set represents the essential message sent from and to the
	application layers. This colour set is defined as a record, which means a tuple
	of elements that can be referred to by name. |MESSAGE| has two elements: the Op
	|OPERATION| and the Message |STRING|. In the WebSocket protocol, both data- and
	control-frames can have messages, although the message part of control-frames
	does not have to be shown to the user. We can also have a list of |MESSAGES| to
	keep them ordered. Lists will be explained in detail later where they are used and
	manipulated in the model.
	
	To be able to connect to a server, its URL needs to be known, and this concept
	is defined by the |URL| color set. It consists of the Protocol (for
	example http), the Host (www.example.com), the Port (for example 80) and the
	Path (for example /home/index.html).
	
	The Client Application can send tokens to the WebSocket layer as a
	|CLIENT_CALL|. This is a union color set, meaning a token can be one of the
	declared identifiers, optionally with an associated colour set. A union colour set is used if a place should be
	able to contain tokens from different colour sets, or if such tokens should be
	handled in the same way at a point in the model. |CLIENT_CALL|
	has two identifiers: |Connect|, used to request a connection to the
	associated |URL|, and |CliSendMsg|, signifying an outbound message from the
	client. And similar to |MESSAGES|, |CLIENT_CALLS| is a list of |CLIENT_CALL|.
	
	When a connection attempt is completed, the result is represented from the
	|CONN_RESULT| color set, which is a simple rebranding of bool values to improve
	readability.
	
	The WebSocket layer needs a way to notify the client application about 
	connection results as well as received messages, which is done with the
	|CLIENT_EVENT| colour set. Like |CLIENT_CALL|, this is a union colour set which
	in this case can be either a |MESSAGE|, or a |CONN_RESULT|, and we similarly
	have a list version |CLIENT_EVENTS|.
	
	The reason we use a single place to transmit both connection information and
	messages between application and WebSocket layers, is that this makes it easier
	to extend the model later if tokens from other color sets need to be passed
	between the modules. The benefit is that we will not have to create extra places for
	this, and simply need to add the new color sets to the releveant color set
	declaration (for example |CLIENT_CALL|).
	
	Both the client and server WebSocket layers send and receive
	|PACKET| tokens, a union of three color sets that will be explained later.
	|PACKET|s are abstract and not fully modeled actual network packets, as this is
	not relevant to how WebSocket works. This color set is also used in a list 
	|PACKETLIST|.
	

\subsection{Client Application}

	\fig{ClientApplication.eps}{The Client Application}{client_app}
	
	This module (\figref{client_app}) is meant to serve as a generic invocation of
	the WebSocket protocol. It also shows the rest of the essential building blocks
	of a CPN model.
	
	Transitions, represented with single border rectangles, are elements that
	represent potential state changes in the model. They are connected to places by
	directed arcs. These arcs can be inscribed with expressions containing
	variables. Variables are declared globally and can only be bound to tokens of
	the colour set they are defined for. A set of variables has been declared for
	the simple coloursets as follows:
	
	\begin{lstlisting}[label=lst:standard_vars,caption=Simple Colourset
	Variables,gobble=1] 
	var u, u1, u2, u3: UNIT;
	var b, b1, b2, b3: BOOL;
	var i, j, k: INT;
	var s, s1, s2, s3: STRING;
	var ss, ss1, ss2: STRINGLIST;
	\end{lstlisting}
	
	Several variables for the same coloursets have been created for convenience in
	the cases where tokens of the same colour set is consumed from different
	places.	If the same variable was used on each arc from two different places,
	both places would need a token with the same value before the transition
	would be enabled.
	
	A transition is said to be enabled when each of its incoming arcs is able to
	bind the variables in its expression to tokens in its corresponding place. As
	an example, the |Convert to URL| transition in \figref{client_app} is enabled
	(signified with a green glowing border), because the incoming arc's expression
	is simply the variable |s|, which can be bound to the ``websocket.com/chat''
	token in the |Target server| place.
		
	When a transition occurs (or is fired), it will consume said tokens from the
	respective places. These variables can then be used to produce new tokens,
	according to the expression(s) inscribed on the output arc(s), which end up in
	the place(s)those arcs point to. To continue the earlier example, the string
	bound to s becomes an argument for the function |parseUrl| (explained further
	down), which produces an |URL| token that then ends up in the |Target URL|
	place.
	
	Other variable declarations will be listed as they are encountered in the
	model. As a general guideline, most variables are named with the first letter
	of its colourset for non-lists, and the same letter plus the letter s for
	lists. On this submodule we have the following variables:
	
	\begin{lstlisting}[label=lst:client_app_vars,caption=Client Application
	Variables,gobble=1]
	var msg: MESSAGE;
	var msgs, msgs2: MESSAGES;
	var cEvents: CLIENT_EVENTS;
	var cCalls: CLIENT_CALLS;
	\end{lstlisting}
	
	The function |parseUrl| converts a string into an |URL| token. Its
	implementation can be seen in \lstref{client_app_parseUrl}. 
	
	\begin{lstlisting}[label=lst:client_app_parseUrl,caption=parseUrl and related
	functions,gobble=1] 
	fun split2 (s, t, i) =
	   (* Recursively scan for character t in string s starting as position i, split if match *)
	   let
	     val ss = String.extract(s, i, NONE) 
	   in 
	     if String.isPrefix t ss then
	       [substring(s, 0, i), 
	       String.extract(s, i + String.size t, NONE)]
	     else split2(s, t, i+1)
	   end
	 fun split (s, t) = 
	   (* Split string s on character c *)
	   if String.isPrefix t s then
	     [String.extract(s, String.size t, NONE)]
	   else if String.isSubstring t s then
	     split2 (s, t, 1)
	   else [s]
	 fun parseUrl (s) = 
	 let
	   val proto'rest = split (s, "://")
	   val proto'rest = 
	     if length proto'rest = 1 
	     then "ws" :: proto'rest 
	     else proto'rest 
	   val pro = List.hd(proto'rest)
	 
	   val host'path = split (List.nth(proto'rest, 1), "/")
	   val pat = if length host'path = 2
	     then "/" ^ List.nth(host'path, 1)
	     else "/"
	   
	   val host'port = split (List.hd(host'path), ":")
	   val hos = List.hd(host'port)
	 
	   val port'default = case pro of
	     "wss" => 443
	     | _ => 80
	   val por = if length host'port = 1 
	   then port'default
	   else let
	     val port'str = List.nth(host'port, 1)
	     val port'int'opt = Int.fromString port'str
	   in
	     Option.getOpt(port'int'opt, port'default)
	   end
	 in
	   {Protocol=pro, 
	     Host=hos, 
	     Port=por, 
	     Path=pat}
	 end;
	\end{lstlisting}
	
	The places at the bottom, |Client Call| and |Client Event| represent the
	interface to the WebSocket layer, and are paired with the corresponding places
	on the Overview, which are also paired with corresponding places in the
	WebSocket Library. The tokens in these places will be mirrored between the
	sub-module and the super-module. The term for this is ports and sockets,
	where the port is in the sub-module and the socket is in the super-module. The
	ports are labeled with port-type tags (In and Out) to signify the direction
	that tokens move (although this has no technical significance in CPN Tools).
	
	
	\subsubsection{Queues}
	
		A lot of the places in this model rely on tokens being consumed in the same
		order they are produced, in other words, the places should behave like queues.
		However, CPN Tools does not have a mechanism specifically for this purpose.
		Instead, to emulate the queue behaviour we need, we use a list of a colourset
		instead of using the actual colourset we want in that place, and use
		operations defined for lists to access the first and last elements of the
		list.
		
		To describe a list in CPN ML, we use square brackets |[ ]|. By themselves
		they represent an empty list. To describe a populated list, we write each
		token inside the brackets separated by commas. An example of this is seen in
		\figref{client_app} on the initial marking for the |Messages to be sent|
		place.
		
		Lists work similarly to other functional programming languages, defined as
		the head, which is the first element, and the tail, which is the list of all
		remaining elements. To access these elements, we use the \lstinline-::-
		operator like this:	\lstinline-head::tail-. Lists are usually processed in a
		tail-recursive mannner. The |^^| operator is a convenience for concatenating
		two lists.
		
		To model a queue with a list, when we want to append an
		element to a queue, we concatenate the queue using the |^^| operator with a
		new list containing only the new element. When we want to take an element from the front
		of the queue, we use the \lstinline-::- operator to bind the head and tail of the list to variables,
		and put only the tail back to the source place. In both cases this means there
		will be two arcs between the place and the transition: one to bind the
		target queue to a variable, and one that adds or removes an element. The two
		arcs can be misleading since information only logically moves one direction.
		To improve readability of the model, these queue operations have one arc
		colored gray, to emphasise the flow direction of information.
		
		Examples of this is seen in \figref{client_app} on the arcs connecting
		|Client Call| and |Client Event| places at the bottom, as well as the
		|Messages to be sent| and |Messages received| places at the top.

	\subsubsection{Program Flow}
	
		After the |Convert to URL| transition has occured, the |Request connection|
		transition should be enabled. If we were to fire this transition, it would
		consume the |URL| token from |Target URL|, bind the value of that token to the
		variable |url|, create a |Connect| identifier (from the |CLIENT_CALL| union
		color set) containing |url|, and add it to the queue in the |Client Call|
		place.
		
		Next, the Client Application waits for a |CONN_RESULT| token and place it in
		the |Conn Result| place. The expressions on the arcs going out from this
		place use the literal values |success| and |fail| instead of variables. If
		the token has the value |succes|, the |Success| transition is enabled, and it
		adds a |UNIT| token to the |Active Connection| place.
		
		If and when the |Active Connection place| has a |UNIT| token, the
		Client Application can start sending and receiving messages. The arc between
		|Active Connection| and |Send data| is a two-way arc, which works like two
		arcs going in opposte directions with the same expression. The result is that
		the transition is only enabled when there is a token available, but the token
		will not be consumed when the transition occurs. 
		
		A sample of messages has been set as the intial value of the |Messages to be sent|
		place, to simulate an example execution of the program.
		
		Finally, looking at the arc from |Receive Data| to |Active Connection|, if the
		Client Application receives a |MESSAGE| where the |OPERATOR| is |CLOSE|,
		nothing is put back into the Active Connection place, and the connection is
		effectively closed.

\subsection{The Client WebSocket Module}

	\fig{ClientWebSocket.eps}{The Client WebSocket Module}{client_wslib}

	This module consists mostly of sub-modules. The only processing
	being done directly on this module is at the top, where messages and
	connection info is separated, and at the bottom, where
	masking of all websocket frames occurs, which the client is required to do by
	the protocol specification. The rest is plumbing between the submodules.
	
	The new coloursets are for HTTP requests and responses, WebSocket frames, and
	packets. They have corresponding variables. 
	
	\begin{lstlisting}[label=lst:client_lib_colset_http,caption=Library colour
	sets - HTTP,gobble=1]
	colset HTTP_VERB = with GET | POST | PUT | DELETE | HEAD;
	colset REQUEST_LINE = record Verb: HTTP_VERB * Path: STRING * Version: STRING;
	colset HEADER = record Key: STRING * Value: STRING;
	colset HEADERS = list HEADER;
	colset HTTPREQ = record RequestLine: REQUEST_LINE *	Headers: HEADERS;
	
	colset RESPONSE_LINE = record Version: STRING * Status: INT * Message: STRING;
	colset HTTPRES = record ResponseLine: RESPONSE_LINE * Headers: HEADERS;
	\end{lstlisting}
	
	The declarations for HTTP requests and responses
	(\lstref{client_lib_colset_http}) are modeled after the  the HTTP 1.1
	standard. An |HTTPREQ|, or HTTP Request, consists of an initial |REQUEST_LINE|
	(following the standard format of an |HTTP_VERB|, a Path and a Version), and a
	number of |HEADERS| (key-value tuples).
	An |HTTPRES|, or HTTP Response, begins with a |RESPONSE_LINE| (consisting of
	the Version, the response Status, and a status Message), with a subsequent
	|HEADERS| list. A real response would usually also have a body, but this is not
	used in the WebSocket protocol, so there is no need to model it.
	
	\begin{lstlisting}[label=lst:client_lib_colset_ws,caption=Library colour
	sets - WebSocket,gobble=1]
	colset BIT= bool with (clear, set);
	colset BYTE = int with 0x00..0xFF;
	colset MASK = list BYTE with 4..4;
	colset MASKING = union Nomask + Mask:MASK;
	
	val opContinuation = 0x0;
	val opText = 0x1;
	val opBinary = 0x2
	val opConnectionClose = 0x8;
	val opPing = 0x9;
	val opPong = 0xA;
	
	colset WSFRAME = record
		Fin: BIT * Rsv1: BIT * Rsv2: BIT * Rsv3: BIT * 
		Opcode: INT * Masked: BIT * Payload_length: INT * 
		Masking_key: MASKING * Payload: STRING;
	
	colset WSFRAMES = list WSFRAME;
	
	colset PACKET = union HttpReq:HTTPREQ + HttpRes:HTTPRES + WsFrame:WSFRAME;
	colset PACKETLIST = list PACKET;
	\end{lstlisting}
	
	The declarations for WebSocket frames (\lstref{client_lib_colset_ws}) have
	been modeled to approximate the actual memory structure of such frames. To this
	end, the colour sets |BIT| and |BYTE| have been created, where |BIT| is a
	relabeling of boolean values, and |BYTE| is an integer range. A |MASK| is a
	list with exactly 4 |BYTE|s. To model the optionality of masks, |MASKING| is
	either |Nomask| or |Mask| with an associated |MASK|.
	
	Next, we declare a number of static values for convenience, which correspond to
	WebSocket frame operation identifiers, termed opcodes. 
	
	According to its specification, the |WSFRAME|, or WebSocket frame, consists
	of four control bits (only the first one is in use to mark final frames), an
	|Opcode| to describe the type of frame, a Masked bit to mark if the frame
	payload is masked, |Payload_length| to declare the number of bytes in the
	payload, an optional |Masking_key|, and the |Payload| itself. 
	
	\begin{lstlisting}[label=lst:client_lib_vars,caption=Library
	variables,gobble=1]
	var wsframe: WSFRAME;
	var wsframes, wsframes2: WSFRAMES;
	var httpreq: HTTPREQ;
	var httpres: HTTPRES;
	var p: PACKET;
	var ps, ps1, ps2: PACKETLIST;
	\end{lstlisting}
	
	Finally, we also have to declare the variables to be used in arc expressions.
	
	Masking of frames is modeled to only set the masking bit and provide a masking
	key. An actual implementation would also apply the mask to the payload
	according to the specification, however this involves applying the
	XOR-operation on each octet in the payload with each octet in the masking key.
	XOR is not defined in CPN ML and would thus have to be modeled manually. Since
	the effects of masking the payload do not change the overall execution of the
	protocol, we opted to omit it completely, and let the presence of a masking key
	abstractly represent that the payload has been masked. The |mask| function is
	thus defined:
	
	 \begin{lstlisting}[label=lst:client_lib_vars,caption=Library
	variables,gobble=1]
	fun randMask() = Mask([
		BYTE.ran(), BYTE.ran(),
		BYTE.ran(),	BYTE.ran()
	]);
  
	fun mask (ws:WSFRAME) = let
		val ws1 = WSFRAME.set_Masked ws set
		val ws2 = WSFRAME.set_Masking_key ws1 (randMask())
	in 
		ws2
	end;
	\end{lstlisting}
     
	\subsubsection{New connection}
		
		\fig{NewConnection.eps}{New Connection submodule}{new_connection}
		
		This module is fairly straightforward. We take an available |URL|
		and create an HTTP request, which is then queued to be sent by the
		client, as well as keeping a copy of the request for validation purposes when
		the response arrives.
		
		The function httpReqFromUrl takes a URL argument and uses it to produce a
		HTTPREQ token with headers according to the WebSocket protocol requirements.
		All the required headers are specified, but optional headers are not, as this
		is a generic implementation. The nonce is statically defined, partly to keep
		it simple and partly to more easily see it during simulation. The Base64 and
		SHA1 functions are similarly basic for the same reasons.
		
		\begin{lstlisting}[label=lst:httpReqFromUrl,caption=httpReqFromUrl,gobble=2]
		fun B64 str = "B64("^str^")";
		fun SHA1 str = "SHA1("^str^")";

		val origin = "http://www.example.com";
		val nonce = "nonce";
		fun httpReqFromUrl (url:URL) = 
          {
            RequestLine={
              Verb=GET, 
              Path=(#Path url),
              Version=httpVersion
            },
            Headers=[
              {Key="Host", Value=(#Host url)},
              {Key="Upgrade", Value="websocket"},
              {Key="Connection", Value="Upgrade"},
              {Key="Sec-WebSocket-Key", Value=(B64 nonce)},
              {Key="Sec-WebSocket-Version", Value="13"},
              {Key="Origin", Value=origin}
            ]
		};
		\end{lstlisting}
		
		
		
	\subsubsection{Process response}
		
		\fig{ProcessResponse.eps}{Process Response submodule}{process_response}
		
		On this module, the transition fires when a |HTTPRES| token is received from
		the server (and the |HTTPREQ| token from earlier is still waiting in its
		place). The boolean variable b is bound to the result from the isResponseValid function,
		which checks if the server's reply is valid and conforms to the WebSocket
		protocol specification.
		
		\begin{lstlisting}[label=lst:isResponseValid,caption=isResponseValid,gobble=2]
		fun isResponseValid (res:HTTPRES, req:HTTPREQ) = let
		  val rline = #ResponseLine res
		  val headers = #Headers res
		  val accepttoken = generateAccept(
		    getHeader("Sec-WebSocket-Key",
		      (#Headers req)))
		in
		  #Status rline = 101 andalso
		  getHeader("Upgrade", headers) 
		    = "websocket" andalso
		  getHeader("Connection", headers) 
		    = "Upgrade" andalso
		  getHeader("Sec-WebSocket-Accept",headers)
		    = accepttoken
		end
		\end{lstlisting}
		
		It sends the result directly to the Client App, and puts a |CONN_OPEN| token
		in the Active Connection place if b is true.
		
	\subsubsection{Wrap and send}
		
		\fig{WrapAndSend.eps}{Wrap And Send submodule}{wrap_and_send}
		
		This module takes new messages, wraps and optionally fragments them in the
		Fragment and queue submodule (explained below), and sends them if there is
		an open connection. If a Close frame is being sent,the
		connection state will be changed to |CONN_CLOSING|, which prevents
		further sending.
		
		\paragraph{Fragment and queue}
			
			\fig{FragmentAndQueue.eps}{Fragment And Queue submodule}{fragment_and_queue}
			
			The top transition on this has two tasks: Making sure we do not process
			CONNECT messages, and filtering data and control frames into different
			places using the isData function. 
			
			\begin{lstlisting}[label=lst:isData,caption=isData,gobble=3]
			fun isData (msg:MESSAGE) =
			(#Op msg = TEXT) orelse
			(#Op msg = BINARY);
			\end{lstlisting}
			
			Control frames should never be fragmented, but data frames could be. This is
			taken care of by the fragment function, which also converts the message into
			a WSFRAME token. The control frames are converted in the same way by the
			wrapmsg function. Both of these functions rely on the wrap function, and
			therefore have to be declared after it.
			
			
			\begin{lstlisting}[label=lst:fragment,caption=wrap wrapmsg and
			fragment,gobble=3]
			fun wrap (opc,payload,fin) = { Fin=fin,
			  Rsv1=clear,
			  Rsv2=clear,
			  Rsv3=clear,
			  Opcode=opc,
			  Masked=clear,
			  Payload_length=(String.size payload),
			  Masking_key= Nomask,
			  Payload=payload
			}
			
			fun wrapmsg (msg:MESSAGE,fin) = 
			  wrap(opSym2Hex(#Op msg), 
			    (#Message msg), fin);
			    
			fun fragment (msg:MESSAGE) = let
			  fun loop (opc, s, acc) = 
			    if (String.size s) > fragSize
			    then loop(
			      opContinuation,
			      String.extract(s,fragSize,NONE),
			      acc^^[wrap(opc, 
			        String.substring(s,0,fragSize), 
			        clear
			      )]
			    )
			    else 
			      acc^^[wrap(opc, s, set)];
			  in 
			    loop(
			      opSym2Hex(#Op msg), 
			      (#Message msg), 
			      [])
			  end;
			\end{lstlisting}
			
			The WSFRAME tokens are then queued randomly one by one from the data queue or
			the control frame queue. This allows control frames to be injected between
			the parts of a fragmented data frame, as required by the WebSocket protocol
			specification. Control frames are prioritised, by the prescence of a
			two-way arc from the Control place to the Queue data transition, inscribed
			with |[]|, which prevents it from being enabled if the list
			in the Control place is not empty. This priorisation is allowed but not
			required by the WebSocket protocol specification, but is included here to
			emphasise that control frames can be sent even between two fragmented data
			frames.
		
	\subsubsection{Unwrap and receive}
		
		\figrot[0.4]{UnwrapAndReceive.eps}{Unwrap And Receive
		submodule}{unwrap_and_receive}
		
		Received WebSocket frames that arrive in the Packet Received place at the
		bottom can take three paths. 
		
		The first is to the left, and happens if the
		connection is in the |CONN_OPEN| state (checked on the arc) and the
		frame is not a close frame (checked in the guard of the Receive transition). The WebSocket
		frame is put in the Received WS Frame place, and if it is a Ping frame, a Pong
		frame is immediately queued for sending with identical message body. 
		
		The received frame is then checked on two points for fragmentation: If the Fin
		bit is set and the opcode is not continuation, it is not part of a fragmented
		message and converted directly to a MESSAGE. If either or both of those
		conditions are not true, tihs is part of a fragmented message and is processed
		in the Defrag submodule (explained below).
		
		The second path a frame can take is to the right, and happens if the
		connection is in the |CONN_OPEN| state (checked on the arc) and the
		frame is a close frame (checked in the guard of the Receive transition). A
		close frame is created and set to be sent, and the
		connection state is changed to |CONN_CLOSED|, since we have both
		received and sent a close frame. Note that the packetlist from Client Send is
		not appended to but instead discarded, because we can not expect the other
		end to process any more frames other than a close frame since it has already
		sent a close frame of its own.
		
		The third path is upwards and happens if the connection state is
		|CONN_CLOSING|, which means a close frame has been sent and we are
		waiting for a reply. Any payload is ignored, and the connection state stays
		the same until a close frame is received, in which case the connection state
		is set to |CONN_CLOSED|. 
		
		\paragraph{Defragmenting fragmented frames}
			
			\fig{Defrag.eps}{Defragment}{defragment}
			
			Frames that are part of a fragmented message should always arrive in a
			specific order, and correspond to each of the transitions on this module. 
			
			If the Fin bit is clear and the opcode is not continuation, this is the first
			frame in the series. A mew |MESSAGE| is created with the opcode and
			payload from the WebSocket frame and put in the Buffer place.
			
			If the Fin bit is clear and the opcode is continuation, this frame belongs in
			the middle of the sequence. The payload is appended to the
			|MESSAGE| in the Buffer place using the |append()|
			function.
			
			\begin{lstlisting}[label=lst:append,caption=append,gobble=3]
			fun append (msg:MESSAGE, s) = 
			  {Op = #Op msg,
			  Message = (#Message msg)^s}
			\end{lstlisting}
			
			If the Fin bit is set and the opcode is continuation, this is the last frame
			of the sequence. We append the payload to the message and put it in the final
			Completed message place. 
			
			Note that the WebSocket protocol does not allow fragmented messages to be
			interleaved, so it can be assumed that consecutive fragments belong to the
			same message. Fragment inerleaving can be defined by subprotocols, but this
			is not relevant to this model.
			
\subsection{Connection}

	\fig{Connection.eps}{The Connection module}{connection}
	
	\figref{connection} shows the connection; a very simple model. The packets that
	come from the Client Send place go to the Server Receive place, and from the
	Server Send place to the Client Receive place. The transportation of data
	between the client and the server as well as establishing and ensuring a
	stable connection is assumed to be taken care of by TCP, as the WebSocket
	protocol specifies, and is not necessary to model in detail to show how WebSocket works. 
	
	The packets are also not converted to pure bits or bytes. This abstraction was
	made since the inner workings of the TCP layer is not relevant to the WebSocket
	Protocol.

\subsection{The Server WebSocket Layer}
	
	\fig{ServerWebSocket.eps}{The Server WebSocket Layer}{server_wslib}
	
	The Server WebSocket layer is very similar to the Client version. The main
	differences are that instead of masking outgoing frames, we are checking
	incoming frames for a mask and unmasking them, and that we are checking for
	incoming connections and replying to them based on what the Server Application
	decides.
	
	The Server Application can send two types of coloursets: |MESSAGE| and
	|CONN_REPLY|. A special colourset has been made to accomodate this,
	|MSG_OR_CONN_REPLY|. 
	
	The submodules that belong to the |Wrap and Send| and the
	|Unwrap and Receive| substitution transitions are the same as the ones for the Client Library. To
	be more precise, the Client Library and Server Library both have instances of
	the same submodule, so that editing the submodule model affects both parent
	modules while simulating them lets them have different states.
	
	\subsubsection{Get Connection Request}
		
		\fig{ConnectionRequest.eps}{Get Connection Request}{connection_req}
		
		This is a very abstract representation of how connection requests are
		received. The library simply sends a ConnRequest token to the app. In a real
		world app, much more info might have been sent, for example IP address and
		possibly authentication info, but for this model it is enough to show that
		some info is being sent, while the details are abstracted away. 

	\subsubsection{Send Connection Response}
		
		\fig{ConnectionResponse.eps}{Send Connection Response}{connection_res}
		
		When the Server Application has decided what to do with an incoming
		connection, it will send a |CONN_REPLY| to the library. If the
		answer is |accept|, we create a |CONN_OPEN| token in the
		Active Connection place and send a HTTP response back to the client, properly
		formatted according to the rules of the WebSocket Protocol. This involves
		generating a Sec-WebSocket-Accept header, which is done with the
		generateAccept() function:
		
		
		\begin{lstlisting}[label=lst:generateAccept,caption=generateAccept,gobble=2]
		fun B64 str =
		  "B64("^str^")";
		
		fun SHA1 str =
		  "SHA1("^str^")";

		fun generateAccept str =
		  B64(SHA1(str^uuid));
		\end{lstlisting}
		
		The two functions B64() \cite{rfc4648} and SHA1() \cite{fips.180-2} are
		abscract versions of the Base64 encoding algorithm and the SHA1 hasing algorithm. We felt it was unneccessary
		to actually implement these for the purpose of this model, and instead decided
		to simply wrap the string argument to show that it had been encoded or hashed.
		
\subsection{Server Application}
	
	\fig{ServerApplication.eps}{The Server Application Module}{server_app}
	
	The Server Application has three tasks: Accept or reject incoming connections,
	and send and retrieve data. The colourset |MSGS_OR_CONN_REPLY| that
	was defined in \lstref{overview_colset} is seen here. It has three
	queued messages, which illustrate the capability of PONG frames to be used as
	a heartbeat (without PING being involved). A real world application would have
	more logic here, but the interface to the library would be the same.
