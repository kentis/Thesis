\chapter{The WebSocket Protocol CPN Model}
\label{chap:background}

The primary case study developed in this thesis for evaluation of the \thename{}
framework is the WebSocket protocol \cite{draft-ietf-hybi-thewebsocketprotocol}.
The WebSocket protocol is designed to provide two-way communication between a
client running untrusted code running in a controlled environment (e.g. a web
browser) to a remote host that has opted-in to communications from the untrusted code.

Section \ref{sec:ws} briefly introduces the basic operation and concepts of the
WebSocket protocol. Section \ref{sec:ws_cpn_model} presents the constructed CPN
model of the WebSocket protocol and introduces the basic conocepts of the CPN
modelling language.

\section{The WebSocket Protocol}\label{sec:ws}
\fig[0.4]{WebSocketSequenceDiagram.pdf}{Sequence Diagram of the WebSocket
protocol}{wssequence}

\figref{wssequence} shows the basic sequence of message exchanges in the
WebSocket protocol.
To establish a connection, a client sends a specially formatted HTTP request to
a server, which replies with a HTTP response. Once the connection is
established, the client and server can then freely send WebSocket message frames
of different types as defined by a list of opcodes, until either endpoint sends
a control frame with the opcode 0x8 for close and optionally data about the
reason for closing. The other endpoint then replies with the same opcode and
reason, and the connection is considered closed.

Conceptually, WebSocket constitues a layer on top of TCP that
does the following:
\begin{itemize}
	\item adds a Web "origin"-based security model for browsers

	\item adds an addressing and protocol naming mechanism to support
	multiple services on one port and multiple host names on one IP
	address;

	\item layers a framing mechanism on top of TCP to get back to the IP
	packet mechanism that TCP is built on, but without length limits and with
	the reliability that TCP provides

	\item includes an additional closing handshake in-band that is designed
	to work in the presence of proxies and other intermediaries
\end{itemize}

Further details on the operation of the WebSocket protocol will be provided in
the next section when presenting the constructed CPN model.

\section{The WebSocket CPN Model} \label{sec:ws_cpn_model}

A CPN model is organised into pages, also called modules. Each module may in
turn contain sub-modules, which provides a way keep complex and/or large models
clear and manageable. The WebSocket model makes full use of this
hierarchy feature, and we describe the WebSocket CPN model in a top-down
manner in the following subsections.

\subsection{Overview}

	
	\figref{overview} shows the top-level \nodename{Overview} module of the
	WebSocket CPN model. It consists of five sub-modules represented by double
	border rectangles called substitution transitions, and several places
	and arcs that connect them, represented by circles and arrows respectively. 
	The semantic details of arcs and arc expressions will be explained when
	presenting the \nodename{Client Application} module.
	\fig{Overview.eps}{Overview module of the WebSocket CPN model}{overview}
	
	The text in the middle of each substitution trasition is the name of that node,
	while the text in the small blue box attached to the bottom is the name of the
	associated sub-module. These names are often the same for a particular
	substitution transition, although module names cannot contain spaces.

	The substitution transitions have been laid out to resemble part of the OSI
	model \cite{osi7}, where \nodename{Client Application} and \nodename{Server
	Application} each correspond to the top two layers Application and
	Presentation, \nodename{Client WebSocket} and \nodename{Server WebSocket}
	correspond to the Session layer, and \nodename{Connection} corresponds to the
	lower layers Transport, Network, Data Link and Physical layers. They will be
	referred to as the application layer, the protocol layer and the network layer
	respectively.
	
	Places are used to represent the state of the modelled system, and can contain
	tokens. Each token can have a data value, termed the token colour. The number
	of tokens and their colours in a specific place is termed the marking of that
	place. Similarly, the tokens in all places in the model together form the 
	marking of the model, and thus represents the state of the system. The number
	of tokens currently in a place is represented by the number in the small green
	circle attached to the place, and the details of the token colours are shown in
	the small green box next to it. Each place also specifies a colour set
	and an initial marking.
	
	All colour sets, variables, symbolic constants and functions have to be
	declared globally for the model. Colour sets are defined with the following syntax:
	
	|colset <name> = <type-specification>;|
	
	A colour set is defined as a range or set of data
	values, and is declared globally for the model using CPN ML. Colour set names
	are always capitalised in this thesis for easier recognition, but any CPN ML
	identifier would be valid.
	A simple example is the colour set definition |colset INT = int;| representing
	the set of all integers.
	A token from this colour set could for example have the value '3'. Similar
	standard declarations are present for |STRING|, |BOOL| and |UNIT|.
	These are examples of simple color sets. 
	
	It is also possible to declare complex color sets, termed compound color
	sets, to form data structures / data types. One such compound type structure is
	the list, used to define an ordered collection of tokens from one color set.
	Colour sets defined in the WebSocket model will be explained as they are
	encountered in each sub-module. The places in the \nodename{Overview} module
	use the colour set definitions provided in \lstref{overview_colset}.
	
	\begin{lstlisting}[label=lst:overview_colset,caption=Overview colour
	sets,gobble=1,tabsize=4,float=tbh] 
	colset OPERATION = with TEXT | BINARY | PING | PONG | CLOSE; 
	colset MESSAGE = record Op: OPERATION *	Message: STRING;
	colset MESSAGES = list MESSAGE;
	
	colset URL = record Protocol: STRING * Host: STRING * 
						Port: INT * Path: STRING;
	colset CLIENT_CALL = union Connect:URL + CliSendMsg:MESSAGE;
	colset CLIENT_CALLS = list CLIENT_CALL;
	
	colset CONN_RESULT = bool with (fail, success);
	colset CLIENT_EVENT = union CliGetMsg:MESSAGE + 
								ConnResult:CONN_RESULT;
	colset CLIENT_EVENTS = list CLIENT_EVENT;
	
	colset CONN_REPLY = bool with (reject, accept);
	colset SERVER_CALL = union SerSendMsg:MESSAGE + 
							   ConnReply:CONN_REPLY;
	colset SERVER_CALLS = list SERVER_CALL;
	
	colset SERVER_EVENT = union SerGetMsg:MESSAGE + 
								ConnRequest:UNIT;
	colset SERVER_EVENTS = list SERVER_EVENT;
	
	colset PACKET = union HttpReq:HTTPREQ + HttpRes:HTTPRES + 
						  WsFrame:WSFRAME;
	colset PACKETLIST = list PACKET;
	\end{lstlisting}
	
	The |OPERATION| colour set is an enumeration that represents the
	different types of messages that can be passed between the application layer
	and the protocol layer. All of these correspond with opcodes used in WebSocket
	frames.
	
	The |MESSAGE| colour set represents the messages sent between the
	application and protcol layers. This colour set is defined as a record, which
	means a tuple of elements that can be referred to by name. |MESSAGE| has two
	elements: the Op |OPERATION| and the Message |STRING|. In the WebSocket
	protocol, both data- and control-frames can have messages, although the message
	part of control-frames does not have to be shown to the user. We also have a
	list of |MESSAGES| to keep them ordered, the use of which will be explained in
	the following subsection.
	
	For the client to connect to a server in the WebSocket protocol, the URL
	of the server needs to be known, and this concept is defined by the |URL| color
	set. It consists of the Protocol (for example http), the Host
	(www.example.com), the Port (for example 80) and the Path (for example
	/home/index.html).
	
	The Client Application can send tokens to the WebSocket layer as a
	|CLIENT_CALL|. This is a union color set, meaning a token can be obtained
	using one of the constructors of the union color set with an argument that
	matches the constructor. A union colour set is typically used if a place should
	be able to contain tokens from different colour sets, or if such tokens should
	be handled in the same way at a  point in the model. |CLIENT_CALL| has two
	constructors: |Connect| (used to request a connection to the associated |URL|),
	and |CliSendMsg| (signifying an outbound message from the client). Similar
	to |MESSAGES|, |CLIENT_CALLS| is a list of |CLIENT_CALL|.
	
	When a connection attempt is completed, the result is represented from the
	|CONN_RESULT| color set, which is a simple rebranding of boolean values
	|fail| (false) and |success| (true) in order to improve readability.
	
	The WebSocket layer needs a way to notify the client application about 
	connection results as well as received messages, which is done with the
	|CLIENT_EVENT| colour set. Like |CLIENT_CALL|, this is a union colour set which
	in this case can be either a |MESSAGE| or a |CONN_RESULT|. We similarly
	have a list version |CLIENT_EVENTS|.
	
	
	The reason we use a single place to transmit both connection information and
	messages between the application and protocol layers, is that this makes it
	easier to extend the model later if tokens from other color sets need to be passed
	between the modules. The benefit is that we will not have to create extra places for
	this in the overview and all connected submodules, and instead simply need to
	add the new color sets to the releveant color set declaration (for example
	|CLIENT_CALL|).
	
	Both the client and server WebSocket modules send and receive
	|PACKET| tokens, a union of three color sets that will be explained later.
	|PACKET|s are abstract and do not fully model actual network packets, as this
	level of detail is not required in order to model the operation of the
	WebSocket protocol.
	This color set is also used in a list |PACKETLIST|.
	

\subsection{Client Application}

	
	This module (\figref{client_app}) is designed to serve as a generic invocation
	of the WebSocket protocol. It also shows the rest of the essential building blocks
	of a CPN model.
	\fig{ClientApplication.eps}{The Client Application Module}{client_app}
	
	Transitions, represented with single bordered rectangles, are elements that
	capture state changes in the model. They are connected to places by
	directed arcs. These arcs can be inscribed with CPN ML expressions containing
	variables. Variables are declared globally and can only be bound to values
	(colours) of the colour set they are defined for. A variable declaration has
	the form:
	
	|var <name> = <color-set>;|
	
	The set of variables declared for the simple coloursets are provided in
	\lstref{standard_vars}, and some of them can be seen in use on arcs in
	\figref{client_app}. 
	
	\begin{lstlisting}[label=lst:standard_vars,caption=Simple Colourset
	Variables,gobble=1,float=h] 
	var u, u1, u2, u3: UNIT;
	var b, b1, b2, b3: BOOL;
	var i, j, k: INT;
	var s, s1, s2, s3: STRING;
	var ss, ss1, ss2: STRINGLIST;
	\end{lstlisting}
	
	The execution of a CPN model consists of enabled transitions removing tokens
	connected from its input places and adding tokens to its output places. In
	order to talk about the enabling of a transition, a binding that assignes
	values to the variables of the transition must be provided. The variables of a
	transition are the variables occurring in the arc expressions of arcs
	connected to the transition. Assigning values to the variables of a transition
	allow the arc expressions to be evaluated in order to determine the value of
	tokens to be removed and added. A binding of a transition is enabled when
	there are tokens available in the input places that match the color sets that
	the variables has been declared for. 

	As an example, the \nodename{Convert to URL} transition in \figref{client_app} is
	enabled (signified with a green glowing border), because the incoming arc's
	expression is simply the variable |s|, which can be bound to the string value
	``websocket.com/chat'' of the token in the \nodename{Target server} place. The variable
	s is then used as an argument for the function |parseUrl| (explained further
	down), which produces an |URL| token that is added to the \nodename{Target URL} place.
	
	Several variables have been created for each colour set for convenience in
	the cases where tokens of the same colour set but with different colours is
	consumed from separate places. If the same variable was used on each arc from
	two separate places, both places would need a token with the same colour
	before the transition would be enabled. 
	
	Variable declarations will be listed as they are encountered in the
	model. \lstref{client_app_vars} shows the remaining variables used in the
	\nodename{Client Appliaction} submodule.
	
	\begin{lstlisting}[label=lst:client_app_vars,caption=Client Application
	Variables,gobble=1]
	var msg: MESSAGE;
	var msgs, msgs2: MESSAGES;
	var cEvents: CLIENT_EVENTS;
	var cCalls: CLIENT_CALLS;
	\end{lstlisting}
	
	The function |parseUrl| converts a string into an |URL| token. Its
	implementation can be seen in \lstref{client_app_parseUrl}. It makes use of a
	|split| function, which splits a string on a given substringand returns the
	result as a list. |parseUrl| will make sure all the parts of the returned |URL|
	token are given values, and provides default values for any part that is
	missing from its string argument. 
	
	\begin{lstlisting}[label=lst:client_app_parseUrl,caption=The parseUrl function,gobble=1,float] 
	 fun parseUrl (s) = let
	   val proto'rest = split (s, "://")
	   val proto'rest = 
	     if length proto'rest = 1 
	     then "ws" :: proto'rest 
	     else proto'rest 
	   val pro = List.hd(proto'rest)
	 
	   val host'path = split (List.nth(proto'rest, 1), "/")
	   val pat = if length host'path = 2
	     then "/" ^ List.nth(host'path, 1)
	     else "/"
	   
	   val host'port = split (List.hd(host'path), ":")
	   val hos = List.hd(host'port)
	 
	   val port'default = case pro of
	     "wss" => 443
	     | _ => 80
	   val por = if length host'port = 1 
	   then port'default
	   else let
	     val port'str = List.nth(host'port, 1)
	     val port'int'opt = Int.fromString port'str
	   in
	     Option.getOpt(port'int'opt, port'default)
	   end
	 in
	   {Protocol=pro, Host=hos, Port=por, Path=pat}
	 end;
	\end{lstlisting}
	
	The places at the bottom of \figref{client_app}, \nodename{Client Call} and
	\nodename{Client Event}  represent the interface to the WebSocket layer, and
	are paired with the accordingly named places on the \nodename{Overview} module,
	which are also paired with corresponding places in the \nodename{Client
	WebSocket} module. The tokens in these places will be mirrored between the
	sub-module and the super-module. The term for this is ports and sockets,
	where the port is in the sub-module and the socket is in the super-module. The
	ports are labeled with port-type tags (In and Out) to signify the direction
	that tokens move (although this has no technical significance in CPN Tools).
	
	
	\subsubsection{Queues}
	
		Under normal execution of a CPN model, tokens in a particular place can be
		consumed in any order. But for the WebSocket CPN model, a number of places
		rely on tokens being consumed in the same order they are produced,
		in other words, the places should behave like queues. 
		However, CPN Tools does not have a mechanism specifically for controlling
		this.
		Instead, to emulate the queue behaviour, we use a list of a colourset
		instead of using the actual colourset we want in that place, and use
		operations defined for lists to access the first and last elements of the
		list. 
		
		To describe a list in CPN ML, we use square brackets |[ ]|, with the empty
		list specified as |[]|. To describe a populated list, we write each token
		inside the brackets separated by commas. An example of this is seen in
		\figref{client_app} on the initial marking for the \nodename{Messages to be sent}
		place. This list consists of records representing WebSocket messages.
		
		Lists work similarly to other functional programming languages, with the
		first element being denoted the head of the list, and the remaining elements
		being denoted the tail of the list. To access these elements, we use the |::|
		list constructor like this: |head::tail|. Lists are usually processed in a
		tail-recursive mannner. The |^^| operator is used for concatenating two lists.
		
		To model a queue with a list, when we want to append an
		element to a queue, we concatenate the queue using the |^^| operator with a
		new list containing only the new element, thus placing the element at the end
		of the queue. When we want to take an element from the front of the queue, we
		use the \lstinline-::- operator to bind the head and tail of the list to
		variables, and put only the tail back to the source place. In both cases this
		means there will be two arcs between the place and the transition: one to
		bind the target queue to a variable, and one that adds or removes an element.
		The two arcs can be confusing since information logically moves in only one
		direction. To improve readability of the model, these queue operations have
		one arc colored gray, to emphasise the flow direction of information.
		
		Examples of this is seen in \figref{client_app} on the arcs connecting
		\nodename{Client Call} and \nodename{Client Event} places at the bottom, as well as the
		\nodename{Messages to be sent} and \nodename{Messages received} places at the top.

	\subsubsection{Control Flow}
	
		After the \nodename{Convert to URL} transition has occured (see \figref{client_app}),
		the \nodename{Request connection} transition will be enabled. If we were to fire this
		transition, it would consume the |URL| token from \nodename{Target URL}, bind
		the value of that token to the variable |url|, create a |Connect| identifier
		(from the |CLIENT_CALL| union color set) containing the |url|, and add it to
		the queue in the \nodename{Client Call} place.
		
		Next, the Client Application waits for a |ConnResult| token to appear on
		place \nodename{Client Event} and place it in the \nodename{Conn Result}
		place. The expressions on the arcs going out from the \nodename{Conn Result}
		place use the literal values |success| and |fail| instead of variables. If  the token has
		the value |succes|, the \nodename{Success} transition is enabled, and it
		adds a |UNIT| token to the \nodename{Active Connection} place. This models
		that the WebSocket connection has now been established.
		
		If and when the \nodename{Active Connection} place has a |UNIT| token, the
		Client Application can start sending and receiving messages. The arc between
		\nodename{Active Connection} and \nodename{Send data} is a two-way arc,
		technically the same as two arcs going in opposite directions with the same
		expression.
		The result is that the transition is only enabled when there is a token
		available, but the token will not be consumed when the transition occurs. 
		
		A sample of messages has been set as the intial marking of the
		\nodename{Messages to be sent} place, to simulate an example execution of the program.
		
		Finally, looking at the arc from \nodename{Receive Data} to \nodename{Active
		Connection}, if the \nodename{Client Application} receives a |MESSAGE| where
		the |OPERATOR| is |CLOSE|, nothing is put back into the \nodename{Active
		Connection} place, and the connection is effectively closed.

\subsection{The Client WebSocket Module}\label{sec:client_ws}

	\fig{ClientWebSocket.eps}{The Client WebSocket Module}{client_wslib}

	This module (shown in \figref{client_wslib}) consists mostly of sub-modules.
	The only processing being done directly on this module is at the top, where
	messages and connection info is separated, and at the bottom, where
	masking of all websocket frames occurs, which the client is required to do by
	the protocol specification. The rest is plumbing between the submodules. 
	
	The \nodename{New connection request} submodule handles creation and sending of
	HTTP requests based on the URL specified by the application. The 
	\nodename{Process response} submodule processes the expected HTTP reply from
	the server. The \nodename{Wrap and send} submodule forms WebSocket frames from
	the messages sent from the client application, and the \nodename{Unwrap and
	receive} submodule does the reverse with WebSocket frames coming from the
	server. All of these submodules will be detailed in the coming sections.
	
	The new coloursets used in this module are for HTTP requests and responses,
	WebSocket frames, and packets. They have corresponding variables. The
	declarations for HTTP requests and responses (\lstref{client_lib_colset_http})
	are modeled after the  the HTTP 1.1 standard.
	
	\begin{lstlisting}[label=lst:client_lib_colset_http,caption=HTTP colour
	sets,gobble=1,float=h]
	colset HTTP_VERB = with GET | POST | PUT | DELETE | HEAD;
	colset REQUEST_LINE = record Verb: HTTP_VERB * Path: STRING * Version: STRING;
	colset HEADER = record Key: STRING * Value: STRING;
	colset HEADERS = list HEADER;
	colset HTTPREQ = record RequestLine: REQUEST_LINE *	Headers: HEADERS;
	
	colset RESPONSE_LINE = record Version: STRING * Status: INT * Message: STRING;
	colset HTTPRES = record ResponseLine: RESPONSE_LINE * Headers: HEADERS;
	\end{lstlisting}
	 
	An |HTTPREQ| (or HTTP Request),	consists of an initial |REQUEST_LINE|
	(following the standard format of an |HTTP_VERB|, a Path, and a Version), and
	a number of |HEADERS| (key-value tuples). An |HTTPRES|, or (HTTP Response),
	begins with a |RESPONSE_LINE| (consisting of the HTTP Version, the response
	Status, and a status Message), with a subsequent |HEADERS| list. A real
	response would usually also have a body, but this is not used in the
	WebSocket protocol, hence this has been abstracted away in the CPN model.
	
	\begin{lstlisting}[label=lst:client_lib_colset_ws,caption=WebSocket colour
	sets,gobble=1,float] 
	colset BIT= bool with (clear, set);
	colset BYTE = int with 0x00..0xFF;
	colset MASK = list BYTE with 4..4;
	colset MASKING = union Nomask + Mask:MASK;
	
	val opContinuation = 0x0;
	val opText = 0x1;
	val opBinary = 0x2
	val opConnectionClose = 0x8;
	val opPing = 0x9;
	val opPong = 0xA;
	
	colset WSFRAME = record
		Fin: BIT * Rsv1: BIT * Rsv2: BIT * Rsv3: BIT * 
		Opcode: INT * Masked: BIT * Payload_length: INT * 
		Masking_key: MASKING * Payload: STRING;
	colset WSFRAMES = list WSFRAME;
	
	colset PACKET = union HttpReq:HTTPREQ + HttpRes:HTTPRES + WsFrame:WSFRAME;
	colset PACKETLIST = list PACKET;
	\end{lstlisting}
	
	The declarations for WebSocket frames (\lstref{client_lib_colset_ws}) have
	been modeled to approximate the actual memory structure of such frames. To this
	end, the colour sets |BIT| and |BYTE| have been created, where |BIT| is a
	relabeling of boolean values, and |BYTE| is an integer range. A |MASK| is a
	list with exactly 4 |BYTE|s. To model that masks are optional, |MASKING| is
	either |Nomask| or |Mask| with an associated |MASK|.
	
	Next, we declare a number of symbolic values for convenience, which correspond
	to WebSocket frame operation identifiers, termed opcodes. 
	
	According to its specification, the |WSFRAME|, or WebSocket frame, consists
	of four control bits (only the first one is in use to mark final frames), an
	|Opcode| to describe the type of frame, a| Masked| bit to mark if the frame
	payload is masked, |Payload_length| to declare the number of bytes in the
	payload, an optional |Masking_key|, and the |Payload| itself. Finally, we
	also have to declare the variables to be used in arc expressions
	(\lstref{client_lib_vars}).
	
	\begin{lstlisting}[label=lst:client_lib_vars,caption=WebSocket
	Module Variables,gobble=1,float=h] 
	var wsframe: WSFRAME;
	var wsframes, wsframes2: WSFRAMES;
	var httpreq: HTTPREQ;
	var httpres: HTTPRES;
	var p: PACKET;
	var ps, ps1, ps2: PACKETLIST;
	\end{lstlisting}
	
	All frames sent from a WebSocket client should mask their payload using a four
	byte masking key.
	The motivation for this is to prevent potential network intermediaries like proxy
	servers from ineterfering with transmission of frames based on their content.
	Masking of frames is modeled to only set the masking bit and provide a
	masking key. A real-world implementation would also apply the mask to the
	payload according to the protocol specification, which involves applying the
	XOR-operation on each octet in the payload with each octet in the masking key.
	XOR is not defined in CPN ML and would thus have to be modeled manually. Since
	the effects of masking the payload do not change the overall execution of the
	protocol, we opted to omit it completely, and let the presence of a masking key
	abstractly represent that the payload has been masked. The |mask| function is
	shown in \lstref{mask_fun}, along with the |randMask| function it uses.
	
	\begin{lstlisting}[label=lst:mask_fun,caption=Masking functions
	,gobble=1,float=h]
	fun randMask() = Mask([BYTE.ran(), BYTE.ran(), BYTE.ran(), BYTE.ran()]);
  
	fun mask (ws:WSFRAME) = let
		val ws1 = WSFRAME.set_Masked ws set
		val ws2 = WSFRAME.set_Masking_key ws1 (randMask())
	in 
		ws2
	end;
	\end{lstlisting}
     
	\subsubsection{New Connection Module}
		
		\fig{NewConnection.eps}{New Connection submodule}{new_connection}
		
		This module is shown in \figref{new_connection}. We take an available |URL| from the
		\nodename{New Connection} place and create an HTTP request, which is then
		queued to be sent by the client, as well as keeping a copy of the request for
		validation purposes when the response arrives.
		
		The function |httpReqFromUrl| (\lstref{httpReqFromUrl}) takes a |URL| argument
		and uses it to produce a |HTTPREQ| token with headers according to the WebSocket
		protocol requirements. 
		All the required headers are specified, but optional headers are not, as this
		is a generic implementation. The values |httpVersion|, |origin| and |nonce|
		have been statically defined, partly for simplification and partly to more
		easily see it during simulation. The |nonce| value should be generated
		randomly under real-world execution, however there should be no side effects
		from abstracting this in our model.

		In a similar vein, the two functions |B64()| and |SHA1()| are
		abstract versions of the Base64 encoding algorithm \cite{rfc4648} and the SHA1
		hasing algorithm \cite{fips.180-2} on which the WebSocket protocol relies. We
		considered it unneccessary to actually implement these for the purpose of
		this model, and instead define them to simply wrap the string argument to show
		that it has been encoded or hashed.
		
		\begin{lstlisting}[label=lst:httpReqFromUrl,caption=httpReqFromUrl,gobble=2,float]
		val httpVersion = "HTTP/1.1";
		val origin = "http://www.example.com";
		val nonce = "nonce";
		
		fun B64 str = "B64("^str^")";
		fun SHA1 str = "SHA1("^str^")";
		
		fun httpReqFromUrl (url:URL) = 
          {
            RequestLine={
              Verb=GET, 
              Path=(#Path url),
              Version=httpVersion
            },
            Headers=[
              {Key="Host", Value=(#Host url)},
              {Key="Upgrade", Value="websocket"},
              {Key="Connection", Value="Upgrade"},
              {Key="Sec-WebSocket-Key", Value=(B64 nonce)},
              {Key="Sec-WebSocket-Version", Value="13"},
              {Key="Origin", Value=origin}
            ]
		};
		\end{lstlisting}
		
		The rationale for the elaborate set of headers is to ensure that only
		correctly implemented WebSocket servers will recognise the request as a
		WebSocket connection request and respond appropriately, while still adhering
		to the HTTP specification so regular webservers can understand the request.
		
		
	\subsubsection{Process Response Module}
		
		\fig{ProcessResponse.eps}{Process Response submodule}{process_response}
		
		On this module, shown in \figref{process_response}, the transition
		\nodename{Process HTTP Response} is enabled when a |HTTPRES| token arrives
		from the server (and the |HTTPREQ| token from earlier is still waiting in the
		place \nodename{Waiting for HTTP Response}). The transition has an
		inscription (termed its guard) where the boolean variable |b| is bound to the
		result from the |isResponseValid| function (\lstref{isResponseValid}), which checks if the
		server's reply is valid and conforms to the WebSocket protocol specification.
		This involves generating a hash token that proves the server has interpreted
		the original request in accordance with the WebSocket specification. The hash
		is computed using |B64| and |SHA1| on a static universally unique identifier
		(UUID) (the same one should be used in every WebSocket implementation) and the
		nonce previously generated by the client for the |Sec-WebSocket-Key| header.
		
		\begin{lstlisting}[label=lst:isResponseValid,caption=isResponseValid,gobble=2,float]
		val uuid = "258EAFA5-E914-47DA-95CA-C5AB0DC85B11";
		fun generateAccept str = B64(SHA1(str^uuid));
		
		fun isResponseValid (res:HTTPRES, req:HTTPREQ) = let
		  val rline = #ResponseLine res
		  val headers = #Headers res
		  val accepttoken = generateAccept(
		    getHeader("Sec-WebSocket-Key", (#Headers req)))
		in
		  #Status rline = 101 andalso
		  getHeader("Upgrade", headers) = "websocket" andalso
		  getHeader("Connection", headers) = "Upgrade" andalso
		  getHeader("Sec-WebSocket-Accept", headers) = accepttoken
		end
		\end{lstlisting}
		
		Since |b| is a boolean variable, we can use it directly to
		notify the \nodename{Client App} through the \nodename{connection} place
		whether the response was valid and thus if the connection was successful or not.
		Additionally, if |b| is |true|, a |CONN_OPEN| token is put in the
		\nodename{Active Connection} place to signify that the connection is open.
		
	\subsubsection{Wrap and Send Module}
		
		\fig{WrapAndSend.eps}{Wrap And Send submodule}{wrap_and_send}
		
		This module (\figref{wrap_and_send}) takes new messages from place
		\nodename{Queue}, wraps them as WebSockert frames and optionally fragments
		them in the \nodename{Fragment and queue} submodule, and sends them if there is an open
		connection as indicated by the token on the \nodename{Connection status}
		place. If a Close frame is being sent, the connection state will be changed to
		|CONN_CLOSING|, which will also prevent sending of subsequent frames.
		
		\paragraph{Fragment and Queue Module}
			
			\fig{FragmentAndQueue.eps}{Fragment And Queue submodule}{fragment_and_queue}
			
			This module is shown in \figref{fragment_and_queue}. The 
			\nodename{Sort control and data} uses the isData function (\lstref{isData}) to filter
			the two types of messages that can be sent with the WebSocket protocol:
			Data (meaning actual information the application wants to send, either
			binary or textual) and Control (ping, pong and close).
			
			\begin{lstlisting}[label=lst:isData,caption=isData,gobble=3,float=h]
			fun isData (msg:MESSAGE) =
			(#Op msg = TEXT) orelse
			(#Op msg = BINARY);
			\end{lstlisting}
			
			Control frames should never be fragmented and can thus be directly wrapped
			with the |wrapmsg| function (\lstref{fragment}).
			Data frames with long payloads should be fragmented. This is taken care of by
			the |fragment| function, which uses an inner recursion to split the payload. 
			Both of these functions employ the |wrap| function (\lstref{fragment},
			which has to be declared before it is used by the other two functions) to
			produce |WSFRAME| tokens.
			
			
			\begin{lstlisting}[label=lst:fragment,caption=wrap wrapmsg and fragment,gobble=3,float] 
			fun wrap (opc,payload,fin) = { 
			  Fin=fin, Rsv1=clear, Rsv2=clear, Rsv3=clear,
			  Opcode=opc, Masked=clear,
			  Payload_length=(String.size payload),
			  Masking_key=Nomask, Payload=payload
			}
			
			fun wrapmsg (msg:MESSAGE,fin) = 
			  wrap(opSym2Hex(#Op msg), 
			    (#Message msg), fin);
			    
			fun fragment (msg:MESSAGE) = let
			  fun loop (opc, s, acc) = 
			    (* Recurse over s, adding ws frames to 
			    accumulator acc and returning it when 
			    remaining s will fit max fragment size,
			    and manage opcode and fin bit *)
			    if (String.size s) > fragSize
			    then loop(
			      opContinuation,
			      String.extract(s,fragSize,NONE),
			      acc^^[wrap(opc, 
			        String.substring(s,0,fragSize), 
			        clear
			      )]
			    )
			    else 
			      acc^^[wrap(opc, s, set)];
			  in 
			    loop(
			      opSym2Hex(#Op msg), 
			      (#Message msg), 
			      [])
			  end;
			\end{lstlisting}
			
			After \nodename{Sort control and data}, the WSFRAME tokens are queued
			one by one into the data queue or the control frame queue depending on its
			type. This allows control frames to be injected between the parts of a
			fragmented data frame, as required by the WebSocket protocol specification.
			Control frames are prioritised, by the presence of a two-way arc from the
			Control place to the Queue data transition, inscribed with |[]|, which
			prevents it from being enabled if the list in the Control place is not
			empty. This priorisation is allowed but not required by the WebSocket
			protocol specification, but is included here to emphasise that control
			frames can be sent between fragmented data frames.
		
	\subsubsection{Unwrap and Receive Module}
		
		\figrot[0.4]{UnwrapAndReceive.eps}{Unwrap And Receive
		submodule}{unwrap_and_receive}
		
		This module, shown in \figref{unwrap_and_receive}, handles reception of
		frames and extracting their payload into a message.
		
		Received WebSocket frames that arrive in the \nodename{Packet Received} place 
		(bottom) can be processed in three different manners:
		
		\begin{itemize}
			\item The first one is modeled by the \nodename{Receive} transition, and
			happens if the connection is in the |CONN_OPEN| state (checked on the arc
			from place \nodename{Connection Status}) and the frame is not a close frame
			(checked in the guard of the Receive transition). The WebSocket frame is put
			in the Received WS Frame place, and if it is a Ping frame, a Pong frame is
			immediately queued for sending with identical message body. This is modeled
			by the arc expression on the arc going to \nodename{Send Pong}.
			
			\item The second manner a frame can be processed is modeled by the
			\nodename{Close} transition, and happens if the connection is in the
			|CONN_OPEN| state (checked on the arc from \nodename{Connection
			Status}) and the frame is a close frame (checked in the guard of the \nodename{Close} transition). A close frame is
			created and set to be sent as response, and the connection state is changed
			to |CONN_CLOSED|, since the client has both received and sent a close
			frame. Note that the packetlist from \nodename{Client Send} bound to the
			variable |ps| is not appended to but instead discarded, because we can not
			expect the other end to process any more frames other than a close frame
			since it has already sent a close frame of its own.
			
			\item The third case happens if the connection state is in
			|CONN_CLOSING|, which means a close frame has been sent and we are
			waiting for a reply. This is modeled by the \nodename{Waiting for Close}
			transition. Any payload is ignored, and the connection state stays the same
			until a close frame is received, in which case the connection state is set to |CONN_CLOSED|. 
		\end{itemize}
		
		Both the second and third situation described above will queue the received
		close frame in the \nodename{Close Frame} place to notify the application, but the payload is
		stripped through the \nodename{Notify App} transition's outgoing arc
		expression as it should not be exposed to the user according to the
		specification.
		
		The received frame is now in the \nodename{Received WS Frame} place. It is now checked
		on two points for fragmentation:
		If the |Fin| bit is set and the |Opcode| is not continuation, it is not part of
		a fragmented message and converted directly to a |MESSAGE| and placed as a
		token on place \nodename{Received}.
		If either or both of those conditions are not true, this is part of a fragmented message and is processed
		in the \nodename{Defrag} submodule associated with the Fragmented
		substitution transition.
		
		\paragraph{Defragmenting fragmented frames.}
			
			\figref{defrag} shows this module. Frames that are part of a fragmented
			message can have its order determined by its fin bit and its opcode.
			
			\fig{Defrag.eps}{Defrag Module}{defrag}
			
			\begin{itemize}
				\item If the |Fin| bit is not set and the |Opcode| is not continuation, this is the
				first frame in the series. A new |MESSAGE| is created in the
				\nodename{Buffer} place with the opcode and payload from the WebSocket frame.
				
				\item If the |Fin| bit is not set and the |Opcode| is continuation, this
				frame belongs in the middle of the sequence. The payload is appended to the
				|MESSAGE| in the \nodename{Buffer} place using the |append()|
				function (\lstref{append}) to concatenate the strings.
				
				\begin{lstlisting}[label=lst:append,caption=append,gobble=3,float=h]
				fun append (msg:MESSAGE, s) = 
				  {Op = #Op msg,
				  Message = (#Message msg)^s}
				\end{lstlisting}
				
				\item If the |Fin| bit is set and the |Opcode| is continuation, this is the
				last frame of the sequence. We append the payload to the message and put it in the final
				\nodename{Complete message} place. 
			\end{itemize}
			
			Note that since the WebSocket protocol does not allow separate fragmented
			messages to be interleaved, and the TCP protocol guarantees preservation of
			order, it can be assumed that consecutive fragment frames belong to the same
			message and are in the correct order.
			Fragment interleaving can be defined by subprotocols, but this is not
			relevant to this model.
			
\subsection{Connection Module}

	\fig{Connection.eps}{The Connection module}{connection}
	
	\figref{connection} shows the connection layer as modeled by the
	\nodename{Connection} module. The packets that come from the \nodename{Client
	Send} place go to the \nodename{Server Receive} place, and from the
	\nodename{Server Send} place to the \nodename{Client Receive} place. It has
	purposely been modeled abstractly, since the transportation of data between the
	client and the server as well as establishing and ensuring a stable
	connection is assumed to be taken care of by TCP as the WebSocket protocol
	specifies, hence is not necessary to model TCP in detail to capture the
	operation of the WebSocket protocol.
	
	The packets are also not converted to pure bits or bytes. This abstraction was
	made since the inner workings of the TCP layer is not relevant to the WebSocket
	Protocol.

\subsection{The Server WebSocket Module}
	
	\fig{ServerWebSocket.eps}{The Server WebSocket Module}{server_wslib}
	
	The \nodename{Server WebSocket} module (\figref{server_wslib}) is very similar
	to the client-side equivalent.
	The main differences are that instead of masking outgoing frames, we are checking
	incoming frames for a mask and unmasking them, and that we are checking for
	incoming connections and replying to them based on what the \nodename{Server
	Application} decides. The |unmask| function is shown in \lstref{unmask}.
	 
	\begin{lstlisting}[label=lst:unmask,caption=unmask,gobble=1,float=h]
	fun unmask (ws:WSFRAME) = let
	   val ws1 = WSFRAME.set_Masked ws clear
	   val ws2 = WSFRAME.set_Masking_key ws1 Nomask
	in 
	   ws2
	end;
	\end{lstlisting}
		
	The \nodename{Wrap and Send} and the
	\nodename{Unwrap and Receive} substitution transitions are bound to the
	same submodules as the ones in the \nodename{Client WebSocket} module. In
	effect this allows us to reuse the submodules for wrapping and unwrapping
	WebSocket frames (since the mechanics for this is exactly the same for the
	client and the server), while during simulation they will use different
	instances of the bound module, and are thereby able to have different states.
	
	\subsubsection{Get Connection Request Module}
		
		\fig{ConnectionRequest.eps}{Get Connection Request}{connection_req}
		
		This module is shown in \figref{connection_req}, and is a very abstract
		representation of how connection requests are received. It checks the incoming request using
		|isRequestValid|, shown in \lstref{isRequestValid}. If the incoming request
		is valid, a simple |UNIT| token is sent to the application via the
		\nodename{Request Token} place. In an implementation, much more data might
		be provided to the application, for example IP address and possibly
		authentication information from headers, but for this model it is sufficient
		to model that some data is being sent, while the details are abstracted away.

		\begin{lstlisting}[label=lst:isRequestValid,caption=isRequestValid,gobble=2,float]
		fun isRequestValid(req:HTTPREQ) = let
            val rline = #RequestLine req
            val headers = #Headers req
          in
            #Verb rline = GET andalso
            getHeader("Upgrade", headers) 
              = "websocket" andalso
            getHeader("Connection", headers) 
              = "Upgrade" andalso
            getHeader("Origin", headers) 
              = origin 
          end
		\end{lstlisting}
		
	\subsubsection{Send Connection Response Module}
		
		\fig{ConnectionResponse.eps}{Send Connection Response}{connection_res}
		
		This module is shown in \figref{connection_res}. When the \nodename{Server
		Application} has decided what to do with an incoming connection, it will send
		a |CONN_REPLY| (see \lstref{overview_colset}) to the \nodename{Server
		Websocket} module.
		If the answer is |accept|, we create a |CONN_OPEN| token in the
		Active Connection place and send a HTTP response back to the client, properly
		formatted according to the speccification of the WebSocket Protocol. As
		discussed in Section \ref{sec:client_ws} under the \nodename{New Connection}
		and \nodename{Process Response} modules, this involves generating a
		Sec-WebSocket-Accept header based on the Sec-WebSocket-Key header form the
		client, and is done to prove this is a WebSocket sever. The header is
		generated with the |generateAccept| function which was explained in
		\lstref{isResponseValid}.
		
\subsection{Server Application}
	
	\fig[0.4]{ServerApplication.eps}{The Server Application Module}{server_app}
	
	The Server Application has three tasks: Accept or reject incoming connections,
	and sending and retrieval of data. The \nodename{To send} place has three
	messages as its initial marking, to illustrate the capability of PONG frames
	to be used as a heartbeat\footnote{A heartbeat is commonly used to explicitly
	declare that a connection is still active.} (without PING being involved).
	Otherwise, the mechanics of sending and receiving messages is the same as in
	the Client Application. While this module has intentionally been modelled to be
	simple and abstract, a real world application would have more logic here, but
	the interface to the protocol layer would be the same.
