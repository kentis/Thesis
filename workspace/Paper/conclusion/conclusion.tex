\chapter{Conclusion and Future Work}
\label{chap:conclusion}

With \thename{}, we have constructed a fully functional proof of concept that
demonstrates the advantages and disadvantages of using ontologies to manage code
generation pragmatics and applying them to a CPN model. As shown in Chapter
\ref{chap:analysis}, this prototype meets all of the requirements in Section
\ref{sec:requirements}. Through our evaluation by using it to annotate CPN
models of different sizes, we have shown that using ontologies gives a highly
expressive means to defining pragmatics as well as providing automatic
validation of the annotated model. However ontology reasoning using CWA yields
decreasing performance as model size increases, to a point that makes our
prototype unusable for the WebSocket CPN model. Still, we believe our approach
has potential, and warrants further optimisation and research.

This thesis also acts as an example of applied Model Driven Engineering. From
modelling the WebSocket Protocol in CPN Tools, to designing the Annotated CPN
Type for ePNK using EMF, the development process has always focused on the
models. 

\thename{} is still just a prototype and has a long way to go to be considered a
finished product. We will outline some of the ideas we had that
we ultimately deemed out of scope for this thesis. 

The performance issues discussed in Chapter \ref{chap:evaluation} need to
be resolved. We believe the first path to investigate should be finding or
developing an ontology reasoner tailored for CWA.

While writing ontologies by hand is possible, a specialised tool for creating
model specific pragmatics would have been ideal.
Such a tool would give simple mechanics using GUI controls for specifying which
model elements a pragmatic can be attached to, and which parameters it has. The
Plugin Manifest editor is a good example of what we have in mind. 

The CPN Type should be further refined to accurately model sub-modules, ports
and sockets. This would enable more advanced pragmatics; for instance the
Principal pragmatic could be specified to only be available for substitution
transitions on top-level modules, and pragmatics could automatically propagate
through linked ports and sockets. This might require changes to ePNK as well.

The structured labels used in the HLPNG Model Type of ePNK could be adapted to
support CPN ML for  the CPN Type. This could potentially allow advanced pragmatics
placement directly inside the expressions.

The CPN Tools importer could be improved by letting the user choose if the
imported model should be based on the CPN Type or the Annotated CPN Type.
Alternatively, there could be a mechanic for ``upgrading'' the Type from CPN to
Annotated CPN.

