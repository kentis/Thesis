\chapter{Evaluation}
\label{chap:evaluation}

In this chapter we will first demonstrate how to use \thename{} by annotating
the WebSocket CPN model from Chapter \ref{chap:background}. We also briefly
show examples from other CPN models of protocols.
We discuss performance of the reasoner engine, and explain some of the
limitations of \thename{}

\section{Annotating the WebSocket CPN Model}
We begin by showing an overview of the full Eclipse window in
\figref{full_overview}. We have already imported the WebSocket CPN model created
with CPN Tools, using the Import Wizard presented in Section
\ref{sec:import_wizard}. We have opened both the tree editor and the
graphical diagram editor, and laid them out side by side. 

\fig[0.32]{eval/first.png}{Eclipse Overview}{full_overview}

The diagram editor cannot precisely replicate the look of CPN Tools, but is
sufficient to easily recognise the intended layout. The elements can also be
manually moved to clean up mistakes made by the Import Wizard. 

At the bottom, the Properties View is open, showing the properties of the
selected transition. Any of the properties can be directly edited here.

To annotate an element with a pragmatic, we right-click it and open the Annotate
With submenu, shown in \figref{place_channel} (a). The menu shows only
pragmatics that are applicable to this particular place, in this case Id and
Channel. By selecting one of the available pragmatics, it will be added as a
child of the place. The final result is seen in \figref{place_channel} (b),
showing both the tree editor and the Properties View.

\figDbl{eval/place_menu.png}{Pragmatics
Choices}{eval/place_Channel.png}{Result}{Annotating Place with Channel
Pragmatic}{place_channel}

In \figref{overview_annotated}, we show the diagram view of the Overview module
with several pragmatics annotations applied. The Client WebSocket and Server
Websocket substitution transitions have been annotated with \pragma{Principal}
to signify that these are physical endpoints of the protocol. The places between
them have the \pragma{Channel} pragmatic, representing 

\fig[0.4]{eval/overview_annotated.png} {Overview Module Annotated}
{overview_annotated}

After creating the model-specific pragmatics ontology, we must load it into the
model. This is done by right-clicking the PragmaticsOntology in the model,
choosing Add Ontology under the Pragmatics menu (\figref{addOntology} (a)), and
selecting the desired ontology document from the project browser
(\figref{addOntology} (b)).

\figDbl[0.45]{AddOntologyMenu.png}{Add Ontology Menu}
{AddOntologyDialog.png}{Add Ontology Dialog}
{Steps for Adding Pragmatics Ontology}{addOntology}

When we now open the Annotate With submenu, we see the new pragmatic is
available. 

\subsection{The Kao-Chow Protocol Model}

\fig[0.7]{eval/kc_overview.pdf}{KC}{kc}

\section{Performance}

We discovered that the Annotate With submenu takes several minutes to appear.
This is due to the axioms that are added to the ontology to enforce closed world
assumption, causing the reasoner engine to take a long time to classify the
ontology. These axioms consist of the axiom stating that owl:Thing is
equivalent to the asserted individuals and the axiom stating that asserted
individuals are different. As a test, we temporarily commented out these two
axioms, and the Annotate With submenu would then show up almost instantly.
However, its contents would not always be correct, for instance the :External
pragmatic would be incorrectly available on all transitions. Another workaround
is to split the model into several files with one page in each file,
which is what we did to be able to annotate the WebSocket CPN model.
Hypothetically it is possible to design pragmatics with constraints spanning 

A delay of more than a couple of seconds on a frequently used part of an
interactive user interface is not acceptable. We found through gradually
constructing a test model that the delay becomes noticeable when the model has
25 elements, giving a delay of about one second. The Kao-Chow model has 96
elements, which requires about 7 seconds to compute the valid pragmatics. The
WebSocket model has 372 elements, and takes several minutes to classify.

An obvious inefficiency in our approach is that we are constantly recreating the
entire ontology representing the current state of the model. It is worth noting
that EMF has an extensive notification framework for reacting to model changes,
and it should be possible to leverage this to efficiently maintain an accurate
ontology of the loaded CPN model. However, the crucial axioms that enforce CWA
cannot be edited in place, and would thus still need to be removed and added
every time the model is changed.

We believe this problem needs to be addressed at the level of the reasoner
engine implementation. Thanks to the design of OWL API, switching reasoners is
fast and simple. However, none of the four reasoner engines we tested (listed in
Section \ref{sec:ontologies}) gave acceptable performance. Still, as a proof of
concept for the usefulness and expressiveness of ontologies, our approach is
sufficient.

\section{Limitations}\label{sec:limitations}

The graphical diagram editor of ePNK has some shortcomings and we do not
consider it fully mature. Arc appearance is not serialised, making it impossible
to accurately copy layout from CPN Tools. Labels are not attached to their
parent elements, and by default are placed at coordinates 0,0. As ePNK is
continually being updated, we look forward to improvements to this part of
the framework.

Accurate modeling of module instancing (the fact that a module can be a
submodule of several modules) is impractical in the current version of ePNK.
This is again due to the fact that merging EMF models is not possible, meaning
we must subclass existing leaf classes for model elements and therefore inherit
their properties. A subclass of RefPlaces could be used to represent either a
port place or a socket place, but either case presents problems as a RefPlaces
can only reference a single other place.

As a port place, it could be on a module that has several instances, so it would
either need to contain a reference for each instance (which is not possible due
to being limited to one) or the module itself would need to be duplicated
(which is impractical from a user perspective since all annotations would have to be
placed once on each module instance). As a socket place, it could be connected
to several submodules (as is the case with the Connection status place in the
WebSocket CPN model) and would need to have a reference to the ports in each of
them.
