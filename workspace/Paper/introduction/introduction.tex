\chapter{Introduction}
\label{chap:introduction}
 
Software engineering is an increasingly complex discipline, with new and
improved technology emerging at a rapid pace. There is no single answer on how
to approach every challenge in modern software engineering, which has lead to
the development of several software development paradigms. A motivation common
to many of them is that software developers have always sought increasing levels
of abstraction.
Today's software development technology is at a level that potentially gives us
means for automatically generating source code from conceptual domain models of
applications, and substantial research is being conducted to create formal
methods for unleashing this potential.


\section{Model Driven Software Engineering}

Models and diagrams have been used in software design for a long time, and have
been standardised with the introduction of Unified Modeling Language (UML)
\cite{umlInfra} and the methods and tools developed for and around it (like
Rational Unified Process \cite{kruchten2004rational}).
However, models largely play a secondary role, primarily playing the role of
design tools and documentation.

Model Driven Engineering (MDE) \cite{kent2002model} has emerged as a new
development methodology, putting the models in the center of the software
development process. Developers design models that serve both as documentation
and as a basis for implementation, and provides a layer of abstraction over
source code. This trend has been touted as a new programming paradigm, in the
same way object-oriented programming was at the time it was conceived.

Using models as the core design element comes with several benefits. It allows
developers to employ different methods of analysis of the models, like verifying
correctness, completeness, finding race conditions, and analysing scalability.
Models may also act as graphical representations of the system, making them more
understandable for people that are not programmers. This is for instance
important when soliciting requirements from customers.

MDE is based on two central concepts:
\begin{itemize}
	\item Domain specific modeling languages (DSML), which are used to formalise
	application structure, behavior, and requirements of specific domains, such as
	financial services, warehouse management, task scheduling, and protocol and
	communication software. A DSML is defined with a model to describe concepts
	of the domain, along with associated semantics and constraints. This is often
	termed a metamodel
	\item Transformation engines and generators, that process models to produce
	various artifacts in an automated manner. Examples of such
	artifacts include documentation, deployment descriptions, alternative
	representations, and source code ranging from system skeletons to complete,
	deployable products.
\end{itemize}

To define a DSML, a language is needed with which to write the
metamodel definition. This language is itself defined with a model, hence termed
the meta-metamodel, and needs to be capable of describing itself. There exist many such
languages, including the Meta-Object Facility (MOF) \cite{mof} created by the
Object Management Group (OMG) to provide the basis for metamodel definition in
OMG’s family of modeling languages. \figref{metamodel}
shows the relationships between the abstraction layers using UML as an example
DSML for the domain of object-oriented software systems. The layers have
been termed M0 to M3 by OMG, although the number of layers is not rigid and can
be different for other DSMLs. 

\fig{MOF.pdf}{The OMG MOF Metamodelling Layers}{metamodel}

One of the central arguments for MDE is automatic source code generation.
Several advantages come from this:  Documentation and
implementation are synchronised, and boilerplate code and automatic testing is
managed, thus enhancing quality, productivity, reliability, maintainability,
portability and reusability.

One modeling language that is being used as a DSML is Petri Nets. 
Petri Nets are a type of directed graph used to model processes,
especially with an asynchronous and/or concurrent nature. Common
example domains are communication networks and protocols, as
well as concurrent programming design.
Petri Nets are very well suited for MDE, as they have robust methods for
computer-aided verification and analysis. 

There exist many extensions to the Petri Net formalism that define additional
constructs or that change or enhance concepts of Petri Nets. One of these
extensions is called Coloured Petri Nets (CPN). The term Coloured comes from
the fact that a token can have a colour from a defined colour set (type),
essentially data values from a set of values. Combined with the capability to
model synchronicity, concurrency and communication that Petri Nets give, the
functional programming language Standard ML \cite{milner1997definition} is used
as a foundation to define colours and colour sets in a compact manner, and to
provide basic data types and manipulation implementation.

A common way of analysing
Petri Nets is called state space exploration, and is a powerful method for
automatic model verification and determination of several properties. This
thesis includes a short introduction to CPN and state space exploration.

%\subsection{CPN Tools}

CPN Tools \cite{cpntools} is a popular graphical
editor for working with CPN models, from construction and simulation, to
analysis via state space exploration. CPN Tools uses the CPN ML language, an
extension of Standard ML, to specify declarations and net inscriptions.


	
A CPN model can accurately model many types of software systems, but cannot
directly be used to generate a software implementation. Research is being
conducted to develop approaches for and demonstrate how to use CPN and other
Petri Net variants to model software and automatically generate source code.


\section{Thesis Aims and Results}\label{sec:requirements}
This thesis focuses on work done by Simonsen \cite{Simonsen2011}, who discusses
some of the challenges in modelling and automatically generating software. His
work is focused on the domain of communication protocols, and uses the Kao-Chow
authentication protocol as an example.
The ideas introduced in the paper sketch a method for annotating CPNs with a
set of code generation pragmatics that describe how model elements relate to and
bind to source code. Simonsen's approach consists of three parts: 
\begin{itemize}
	\item Annotate the CPN protocol model with pragmatics which bind the model
	entities to program concepts,
	\item Create a platform	model that describes how to implement specific
	constructs for a particular platform,
	\item Create a configuration model for capturing implementation details for a
	particular protocol model.
\end{itemize}  
\fig[1]{KaoChowTopLevel.pdf}{Top level module of the Kao-Chow model}{kaochow}

\figref{kaochow} shows the top level module of the Kao-Chow CPN model from
\cite{Simonsen2011}, with pragmatics enclosed in \textless\textless and
\textgreater\textgreater. These pragmatics have been inserted manually using
auxiliary labels and workarounds to avoid CPN ML syntax errors. This approach is
cumbersome, and there is a need for developing specialised tool support for this
purpose.

Based on the very initial ideas of \cite{Simonsen2011}, the aim of this thesis
is to investigate how to support annotation of CPN models with code generation
pragmatics in a flexible and model-centric manner. By model-centric we mean that
annotation of CPN models should be closely integrated with editing of CPN model
elements.

The research method used to answer this question is to create a prototype
application framework, and evaluate it by annotating a selected set of CPN
models of communication protocols, including a detailed case study on the
application of the WebSocket protocol. The resulting application has been named
\thename{}. This name was chosen to resemble the naming format of other tools that
are built around CPN Tools, including Access/CPN and Design/CPN.

The requirements for the prototype can be divided into four main items:

\begin{itemize} 
	\item Importing CPN Tools models. CPN Tools is one of the best applications
	available for constructing and analysing CPN models, and we wish to
	support models created in CPN Tools.

	\item Annotate model with pragmatics. The prototype should assist the user by
	providing only pragmatics that are applicable to the selected model element.
		
	\item Loading sets of domain-specific pragmatics to make available for the
	model. This is to reduce cluttering and potentially avoid performance
	reductions.
	
	\item Define set(s) of model-specific pragmatics while annotating the model. 

\end{itemize}

The code generation pragmatics (or just ``pragmatics'') in the approach
developed in this thesis are categorised into three classes.
\begin{description}
	\item[General pragmatics] are used to define protocol entities, communication
	channels, external method calls and API entry points for operations like
	establishing a connection, and sending or receiving data.
	\item[Domain specific pragmatics] are pragmatics that apply to all (or many)
	protocols within a particular domain. An example is security protocols, where
	examples of domain specific pragmatics relate to operations such as
	encryption, decryption, and nonce generation.
	\item[Model specific pragmatics] apply only to the specific model instances in
	which they are defined, and are used to label concepts unique to that model. 
\end{description}

\thename{} builds on the ePNK framework \cite{kindler2011epnk}, which uses the
Eclipse Modeling Framework (EMF) to provide an extensible platform for working
with CPN models. The prototype is designed as a plugin for the Eclipse platform,
and can import models created by CPN Tools. It lets the user annotate the models
with pragmatics through a tree editor. Pragmatics are defined using ontologies,
which gives advanced and expressive semantics for classifying pragmatics and
dynamically deduce appropriate pragmatics for individual model elements. These
ontologies can be dynamically loaded into models, which lets the user write
their own ontology containing model specific pragmatics.

\section{Related Work}

Kristensen and Westergaard \cite{kristensen2010automatic} examine challenges of
using CPN for automatic code generation, and propose a new Petri Net type called
Process-Partitioned CPNs. They demonstrate and evaluate it by
designing an implementation for the Dynamic MANET On-demand (DYMO) routing
protocol.

Mortensen \cite{mortensen2000automatic} presented an extension to the Design/CPN
tool to support automatic implementation of systems by reusing the model
simulation algorithm, thus eliminating the usual manual implementation phase.
They demonstrate the tool by implementing an access control system, and
evaluate benefits of the model architecture.

Lassen and Tjell \cite{lassen2010automatic}  present a method for developing
Java applications from Coloured Control Flow Nets (CCFN), a specialised type of
CPN. CCFN forces the modeler to describe the system in an
imperative manner, making it easier to automatically map to Java code.

\section{Thesis Organisation}
The thesis is organised as follows:

\begin{description}
\item[Chapter~\ref{chap:background}:~\nameref{chap:background}.] Provides a
description of the WebSocket protocol, the primary case study used in this
thesis. The chapter gives an introduction to Coloured Petri Nets (CPNs) and the
CPN Tools application used to create the models, combined with a detailed
presentation of the CPN model of the WebSocket protocol produced as part of this
thesis.
\item[Chapter~\ref{chap:statespace}:~\nameref{chap:statespace}.] Gives an
introduction to state space analysis of CPN models, and an example of how to
apply it using the model of the WebSocket Protocol. This validates that the
constructed CPN model of the WebSocket protocol is behaviourally correct.
\item [Chapter~\ref{chap:technology}:~\nameref{chap:technology}.] \thename{} is built
on top of a number of software frameworks and technologies.
This chapter gives an introduction to these as well as the reasons for choosing
them: The Eclipse Platform and its modules; the Eclipse Modeling Framework; the ePNK
framework (which makes up the foundation of \thename{}); Access/CPN (the engine
used to import models from CPN Tools). This chapter also provides an
introduction to ontologies, the format used to specify pragmatics classes.
\item [Chapter~\ref{chap:analysis}:~\nameref{chap:analysis}.] Gives a
discussion and detailing of our solutions for the requirements described
earlier. We start by showing the ontologies that define CPN and Pragmatics, as
well as defining the General Pragmatics. We give details of the ePNK Petri Net
Type Definition for Coloured Petri Nets and Annotated Coloured Petri Nets, and
the mechanic for loading domain and model specific pragmatics into
models. We show how models created with CPN Tools are imported and converted to
ePNK models.
We describe the algorithm used to determine appropriate pragmatics for a
selected model element by using an ontology reasoner, and finally how the user
can write their own model specific pragmatics sets.
\item [Chapter~\ref{chap:evaluation}:~\nameref{chap:evaluation}.] Here we
evaluate \thename{} by annotating various CPN models of protocols, including
the WebSocket CPN model. We also discuss performance compared between the
models, as well as our evaluation of the software technologies used.
\item [Chapter~\ref{chap:conclusion}:~\nameref{chap:conclusion}.] Contains a
summary of the results of the thesis, as well as personal experiences,
limitations and suggested focus of future work.
\end{description} 


The reader is assumed to be familiar with Java programming, the basics of
functional programming languages, and the TCP/IP Protocol Suite. Some basic
knowledge of Petri Nets is also an advantage, but not a strict requirement as we
briefly introduce the basic constructs of the CPN modeling language as part of
describing the WebSocket protocol in Chapter~\ref{chap:background}.
