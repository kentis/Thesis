\chapter{State space analysis of the CPN Web Socket Protocol}
\label{chap:statespace}

One of the advantages about Coloured Petri Nets is state space analysis, which
can tell you about a model's behavioural properties, and can help you locate
errors and increase confidence in the correctness of the model.

\section{State space generation}
A state space is a directed graph where a node represents a reachable marking (a
state) and an arc represents an occurring binding element (a transition firing
with specific values bound to its variables). CPN Tools generates the state
space using Breadth-first-search. 

Once generated, the state space can be visualised directly in CPN Tools.
Starting with the node for the initial state, one can pick a node and show all
nodes that are reachable from it, and in this way explore the state space
visually. This can be very tedious and unmanageable for complex state spaces,
though, and instead it is usually better to use automation to analyse the state
space.

	\subsection{Strongly Connected Component graph}
	In graph theory, a strongly connected component (SCC) is a maximal subgraph
	where all nodes are reachable from each other. A SCC graph consists of each
	SCC in the graph connected with arcs that go between nodes that belong to
	different SCCs. Such a SCC graph is acyclic. A trivial SCC contains only one
	node.
	
	By calculating the SCC graph of the state space, some of the further
	analysis becomes simpler and faster, such as determining reachability and
	looping.

	\subsection{Limitations}
		\subsubsection{State space explosion}
		The biggest drawback of state space analysis is the potential size a state space
		can achieve. The number of nodes and arcs increase with the number of starting
		system parameters (initial marking colors), usually in an exponential relationship. 
		
		This can be remedied by picking smaller configurations,  that encapsulate
		different parts of the system. This was necessary with the WebSocket Protocol
		model, as the state space took too long to generate.
		
		A variant of this is situations where colour tokens can be generated an
		unlimited amount of times, making the state space infinite. This can be
		remedied by modifying the net to limit the number of simultaneous tokens in
		the offending place.
		 
		\subsubsection{Non-deterministic behaviour}
		A model that incorporates random values is not fit for computing a state
		space, as the number of possible bindings are arbitrary for a given state
		depending on the possible random values. Some values might not get used. Other
		times there is an infinite number of possible values (like a floating point
		number).
		
		This can sometimes be alleviated by changing the behaviour to be
		deterministic, for example by replacing the random function with a small color
		set, such as an index or an integer with bounds. This lets the state space
		generator create bindings for every possible value. 
		
		For the WebSocket Protocol model, this was a problem for the masking key in
		WebSocket frames, which is supposed to be a random 4-byte string. The
		randomisation function was simply changed to always return four zeros.

\section{State space report}
Once the state space has been generated, CPN Tools lets the user save a report
as a text document. The report is split into parts that each describe different
aspects about the state space.

To explain each section of the report, a simple report has been generated where
no messages are set to be sent. Thus, the only thing that will happen is that a
connection will be established.
	
	\subsection{Statistics}
	The first section describes general statistics about the state space.
	\begin{lstlisting}[language={}]

  State Space
     Nodes:  17
     Arcs:   16
     Secs:   0
     Status: Full

  Scc Graph
     Nodes:  17
     Arcs:   16
     Secs:   0

	\end{lstlisting}
	This state space has 17 possible markings. We see that it takes 16 transitions
	to fully establish a connection.
	
	There is one more node than there are arcs, which means this graph is a tree. The Secs
	field shows that it took less than one second to calculate this state space,
	while the Status field tells whether the report is generated from a partial or
	full state space.
	
	We also see that the Scc Graph has the same number of nodes and arcs, meaning
	that there are no loops in the state space.
	
	\subsection{Boundedness Properties}
	This section describes the smallest and biggest possible number of markings for
	each place in the model, as well as the actual markings these places can have.
	The text has been truncated for readability.
	\begin{lstlisting}[language={}]
  Best Integer Bounds
                             Upper      Lower
     ClientApplication'Active_Connection 1
                             1          0
     ClientApplication'Conn_Result 1
                             1          0
     ClientApplication'Connection_failed 1
                             0          0
     ClientApplication'Messages_received 1
                             1          1
     ClientApplication'Messages_to_be_sent 1
                             0          0
     .....

  Best Upper Multi-set Bounds
     ClientApplication'Active_Connection 1
                         1`()
     ClientApplication'Conn_Result 1
                         1`success
     ClientApplication'Connection_failed 1
                         empty
     ClientApplication'Messages_received 1
                         1`[]
     ClientApplication'Messages_to_be_sent 1
                         empty
     .....
     ClientWebSocket'Connection_status 1
                         1`CONN_OPEN
     .....
     ServerWebSocket'Connection_Status 1
                         1`CONN_OPEN
     .....

  Best Lower Multi-set Bounds
     ClientApplication'Active_Connection 1
                         empty
     ClientApplication'Conn_Result 1
                         empty
     ClientApplication'Connection_failed 1
                         empty
     ClientApplication'Messages_received 1
                         1`[]
     ClientApplication'Messages_to_be_sent 1
                         empty
     .....
     ClientWebSocket'Connection_status 1
                         empty
     .....
     ServerWebSocket'Connection_Status 1
                         empty
     .....
	\end{lstlisting}
	
	Many places show a lower and upper bound of 1. This shows a weakness
	in the approach of using lists to facilitate ordered processing of tokens: We
	can't see the actual number of tokens that are in the place, because
	technically there is just a list there. 
	
	Apart from that, we see that both the client and the server has an open
	connection at some point.
	
	\subsection{Home Properties}
	This section shows all home markings. A home marking is a marking that can
	always be reached no matter where we are in the state space.
	\begin{lstlisting}[language={}]
  Home Markings
     [17]
	\end{lstlisting}
	We see that there
	is one such marking at node 17. From earlier we know that the state space is a
	tree, and if this node is always reachable it must be a leaf and all the other nodes
	must be in a chain. This tells us that there is only one possible sequence of
	transitions to establish a connection. We can then confidently say that this
	works correctly.
	
	\subsection{Liveness Properties}
	This section . Some of the transitions have ben elided for readability.
	\begin{lstlisting}[language={}]
  Dead Markings
     [17]


  Dead Transition Instances
     ClientApplication'Fail 1
     ClientApplication'Receive_data 1
     ClientApplication'Send_data 1
     ClientWebSocket'Filter_messages 1
     .....

  Live Transition Instances
     None
	\end{lstlisting}
	A dead marking is a marking where no other markings can be reached. In other
	words, there are no transitions for which there are valid bindings, and the
	system is effectivley stopped. For our example, our home marking is also a dead
	marking, confirming that this is a leaf node in the tree. 
	
	We also get a listing of dead transition instances, which are transitions that
	never have any valid bindings and are thus never fired. This is expected for
	many of the transitions in this example, since we're not sending any kind of
	message.
	
	Last, there are live transition instances. A transition t is
	live if we from any reachable marking can find an occurrence sequence containing t.
	Our example has no such transition, which follows trivially from the fact that
	there is a dead marking.

	\subsection{Fairness Properties}
	This section does not apply to our model since it contains no loops. The report
	only has this to say:
	\begin{lstlisting}[language={}]
	No infinite occurrence sequences.
	\end{lstlisting}
	We will not explain more about this, and instead refer you to \cite{cpn_book}
	chapter 7 for more information.
	
\section{TODO:} 
Skrive om resten av analysene.

Flette inn dette:

\subsection{Error discovery}
An error in the model was discovered this way, in the Unwrap and Receive module,
where the pong reply was adding a new list instead of appending to the old one
in outgoing messages.
