\chapter{Technology and Foundations}
\label{chap:technology}

One of the fundamental decisions that had to be made for the work conducted
in this thesis was whether to base \thename{} on an existing platform or to
create a new platform from scratch. In this chapter we describe the
reasoning behind the choices made, taking into account the requirements stated
in Chapter 1, and we give an overview of the technologies that have been
selected.

\section{Representing CPN Models}
	
	A central design decision is how to represent the CPN models. A simple
	but easy way of manipulating a CPN model is by representing it as a tree, with
	modules as nodes, and places, transitions and arcs as child nodes of pages with
	properties describing how they are connected in the CPN model being
	represented. Simple tree editors are a feature of most GUI toolkits,
	but we realised early that creating a CPN model editor from scratch would take
	much longer than adapting an existing platform.

	There already exist many complete implementations of Petri Net tools in
	different languages and toolkits, but few of them are open source, or written
	with extensibility in mind. If we were to base \thename{} on an existing
	platform, it would have to be open and extendable. 
	
	To narrow our search, we limited our options to solutions in
	languages we had experience with: Java, C++/Qt, and Ruby. Java is a popular
	language, and a library called Access/CPN \cite{accesscpn} (a part of the CPN
	Tools project) can parse CPN Tools files, and represent the model using Java
	objects.
	
	The ePNK framework \cite{kindler2011epnk}, an extendable framework for working
	with Petri Nets in a graphical manner, and that makes it possible to specify
	your own Petri Net type. It is built on the Eclipse Modeling Framework (EMF)
	\cite{steinberg2008emf} (which Access/CPN is also built on). 

	A central part of representing the CPM model is also to devise a means for
	representing the code generation pragmatics that can be attached to CPN model
	elements. It was suggested to take an ontology-based approach, and we selected
	the reference Java implementation OWL API \cite{horridge2009owl} to define and work
	with ontologies, together with the HermiT ontology reasoner
	\cite{shearer2008hermit}.
	
	\fig[0.42]{AppOverviewDiagram.pdf}{Application Overview Diagram}{app_overview}
	
	\figref{app_overview} shows the different elements that make up \thename{}.
	The elements with a thick border are the ones created as part of this thesis,
	while the rest represent existing solutions used and built upon. These will
	be described in the following sections. 

\section{The Eclipse Platform}
The Eclipse Rich Client Platform (RCP) \cite{eclipse} is an open source,
cross-platform, polyglot integrated development tools platform.
Its plugin framework makes it highly extendable and customisable, and
especially makes it easy for developers to quickly create tools and
solutions ranging from small custom macros, to advanced editors, to complete
applications. The Eclipse Platform is a part of the Eclipse Project, a community
for incubating and developing open source projects. 

\fig{EclipsePlatformDiagram.pdf}{The Eclipse RCP}{eclipse_rcp}

The architecture of the Eclipse RCP shown in \figref{eclipse_rcp}. We will
give a short explanation of each element:

\begin{itemize}
	\item At the bottom of this we have the Platform Runtime, based on the OSGi
	framework \cite{alliance2007osgi}, which provides the fundamental plugin
	architecture.

	\item The Standard Widget Toolkit (SWT) gives efficient and portable access to
	the user-interface facilities of the underlying operating systems on which it is
	implemented. JFace is a user interface framework built on SWT. The Workbench
	is built using these two frameworks to provide a scalable multi-window editing
	environment.
	
	\item The Workspace defines API for creating and managing resources 
	(pro-jects, files, and folders) that are produced by tools and kept in the file
	system.
	
	\item The Team plugin is a foundation for collaboration and versioning systems.
	It unifies many operations that are common between version control systems.
	
	\item The Help plugin is a web-app-based help system that supports dynamic
	content.
\end{itemize}
	
There are other utilities as well, like search tools, build configuration, and
the update manager which keeps plugins up to date and handles installation of
new plugins. Together these plugins form a basic generic IDE.

Plugins are the building blocks
of Eclipse, and there exists a wide range of plugins that can specialise the
environment for a particular programming language and/or type of application,
as well as add tools, functionality and services. For example, this thesis was
written in \LaTeX{} using the Texlipse plugin, and managed with the Git version
control system through the EGit plugin.

Plugins connect to each other through the so-called extension points they
declare, which includes a schema to define requirements and instructions.
Where one plugin declares an extension point, another plugin can extend that
point according to its defined schema.

It is possible to package Eclipse with sets of plugins to form custom
distributions of Eclipse that are tailored for specific environments and
programming languages. The principal Eclipse distribution is the Eclipse Java
IDE, which is one of the most popular tools for developing Java applications,
from small desktop applications, to mobile apps for Android, to web
applications, to enterprise-scale solutions. Another examples is Aptana Studio,
aimed at Ruby on Rails and PHP development. 

Publishing a custom plugin is simple. By packaging it using specialised tools
in Eclipse (that themselves are plugins) and placing it on a regular web
server, anyone can add the web server URL to the update manager in Eclipse, and
it will let you download and install it directly, as well as enabling update
notifications.


\section{ePNK: Petri Net modeling framework}
\label{sec:epnk}
ePNK is an Eclipse plugin both for working with standard Petri Net models, and a
platform for creating new tools for specialised Petri Net types, which is
exactly what we need for our annotated CPN type. It uses EMF and GMF
(described later) to work with the Petri Net models and provide generic
editors for custom Petri Net variants.

There are several reasons why ePNK is a good choice as a foundation for \thename{}:
\begin{itemize}
	\item It saves models using the ISO/IEC 15909-2 \cite{ISO-15909-2} standard
	file format Petri Net Markup Language (PNML),
	\item It is currently actively developed by researchers in the model-based
	software engineering research group at the Danish Technology University,
	\item It is designed to be generic and easily extendable by creating new model
	types, and
	\item It includes both a tree editor and a graphical editor for CPN models,
	provided through GMF.
\end{itemize}

Below we provide a description of key concepts in ePNK, and for this purpose we
will use the following terminology:
\begin{description}
	\item[Type] - An ePNK Model Type Definition is an extension to ePNK that
	declares and implements a particular Petri Net extension.
	\item[Type Model] - The EMF model defining a Type (as defined above). In MDA
	terms, it corresponds to the metamodel.
	\item[Model] - A model created by a user as an instance of a Type Model.
\end{description}
 
A standard Petri Net Model created in ePNK is initially only defined by the PNML
Core Model Type. Its Type Model is shown in \figref{pnmlcoremodel}. This Type is
intended to be generic, and only defines the basic classes that most Petri Net variants
contain, like Pages (modules), Places, Transitions and Arcs. The only constraint
defined is that an arc must go between two nodes on the same page.
\fig[0.4]{PNMLCoreModel.pdf}{PNML Core Model}{pnmlcoremodel}

The user can choose to extend a Model with features and capabilities of a more 
advanced Petri Net Type. This is done by adding a Type as a child of the Petri Net
Document node in the Model. Only one Type can exist in a Petri Net.

Once a Type is added, ePNK will use class reflection to dynamically load
any associated plugin(s), and the menus for adding new objects to the model will
include any new classes and functionality that the Type defines.

In addition to the PNML Core Model Type, ePNK includes definitions for
two subtypes of Petri Nets. The first is P/T-Nets (Place/Transition Nets), which
expand on the Core Model Type with a few key items: initial markings for places
as integers, inscriptions on arcs, and constraining arcs to only go between a place
and a transition.

The second type included with ePNK is High level Petri Net Graphs (HLPNG). This
type adds Structured Labels which are used to represent model declarations, initial
markings, arc expressions and transition guards. These are parsed and validated
using a syntax that is inspired from (but not the same as or compatible with)
CPN ML from CPN Tools. It is possible to write invalid data in these labels
and still save the document, as they will only be marked as invalid by the
editor to inform the user.

Neither of these two types conform exactly to the Coloured Petri Nets created by
CPN Tools. HLPNG comes close, but is missing a few things like ports and sockets
(RefPlaces can emulate this), and substitution transitions. Also, the
structured labels of HLPNG are not compatible with the CPN ML syntax from CPN
Tools, and for our prototype, these structured labels are not necessary with regard to
pragmatics, as we do not define any pragmatics for this prototype that depend on
this level of detail. This might be desired in a future version, where for
example  pragmatics are available depending on things like the colour set of a
place or the variables on an arc, but initially this is considered to be outside the
scope of this thesis; simple string representations of inscriptions are
sufficient.

In light of this, our decision was therefore to develop our own Petri Net type
that matches the structure of CPN Tools models as well as supporting annotation
with pragmatics, and we present this Type as part of the next chapter.

\section{Eclipse Modeling Framework}
EMF is an open source framework for Model Driven Architecture (MDA) in Java. It
is an Eclipse plugin that is part of the Eclipse Platform. MDA is an
industry architecture proposed by the OMG that aims to unify some of the
industry best practices in software architecture, modeling, metadata management,
and software transformation technologies that allow a user to develop a
modeling specification once and target multiple technology implementations by
using precise transformations/mappings. 

EMF is an example of the use of MDA to enable creation of a UML model
representation of a tool or application and to use this model to automatically
generate some or all of the Java interface, implementation, as well as any XML
serialization for the modelled objects.
Other generated artifacts include a set of adapter classes that enable
viewing and command-based editing of models, and a basic editor. This serves as
the foundation used to build ePNK, including CPN model editing and PNML
serialisation.

	\subsection{Graphical Modeling Framework}
	GMF builds on EMF to provide tools to implement more advanced graphical viewing
	and editing of models. It works by creation of model transformations that use
	the metamodels created with EMF to generate implementations of graphical views and
	editors that plug in to the Eclipse workbench. It is used by ePNK to generate
	its graphical diagram editor.
	
\section{Access/CPN: Java interface for CPN Tools}
CPN Tools has a sister project called Access/CPN. This is an
EMF-based tool to parse .cpn files (files created with CPN Tools) and represent
them as an EMF-model. The .cpn files saved by CPN Tools are XML-based, which makes them
easy to parse. However, having an existing solution for this is preferable, as
we can rely on it to keep up to date with new versions of CPN Tools.

The model definition used by Access/CPN is very similar to that of ePNK. This
will be discussed more in detail in the next chapter.

\section{Ontologies: OWL 2 and OWL API}\label{sec:ontologies}

Ontologies, as they relate to information science, are a way to formally
represent and structure information and knowledge within a domain as a set of
concepts. Essentially, this is done by defining classes that have properties,
relations and constraints, and then describing individuals and known information
about them. 

There is a lot of ongoing research on this subject, especially in the context of
the Semantic Web \cite{bernerslee2001semantic}, that is extending web pages to
provide meta-information about the content they contain and enabling software to
understand its meaning and draw conclusions that are not explicitly asserted.
Use of ontologies is also prevalent in other fields, like in medicine to
describe and relate symptoms and diseases.

The OWL 2 Web Ontology Language \cite{owl2-overview} is the World Wide Web
Consortium (W3C) recommended standard for representing ontologies.  The primary
exchange syntax for OWL 2 is RDF/XML \cite{rdf-xml}. There also exist
other syntaxes, like Manchester syntax which improves readability, and
Functional syntax which emphasises formal structure.

To give a clearer impression of ontologies, consider this small example
ontology:
Person is a class. Man and Woman are subclasses of Person. Parent is also a
subclass of Person, and can be described with the property hasChild with values
of type Person. Mother is a class equivalent to the intersection of Woman and
Parent. Anne and Bob are individuals, and we assert that Anne is
a Woman and that Anne hasChild Bob.

The power of ontologies lies in the potential to use a reasoner engine to
explore and describe the domain, draw conclusions, and infer implicit facts.
From the above example, we can infer that Bob is a Person, and Anne is a Parent,
from the definition of the hasChild property. Thus we can further infer that
Anne is also a Mother, since she is both a Woman and a Parent.

OWL~2 uses an open world assumption, meaning that a fact does not have to be
explicitly asserted for it to be true. We discuss the consequences of this more
in the next chapter.

The code generation pragmatics are defined as OWL 2 ontologies, primarily using
Functional syntax. Together with an ontology describing the structure of CPN
models, this allows us to describe pragmatics using the full expressiveness of
OWL 2, and enables complex restrictions to be defined for placement of
pragmatics on model elements. This acts as a basis to enable the \thename{}
model editor to determine appropriate pragmatics for selected model elements, as
well as for model validation. 

To parse these ontologies, we use OWL API \cite{horridge2009owl}, the reference Java
implementation for OWL 2, which is capable of reading ontologies in any of the
standard OWL 2 syntaxes, as well as programmatically creating and modifying
ontologies.
Implementation details are given in the next chapter.

There exists many reasoner engine implementations, with some that support OWL
API. Dentler et al. \cite{dentler2011comparison} highlight the different
advantages and weaknesses of several such reasoners.
We give our own evaluation of some of them in context of \thename{} in Chapter
\ref{chap:evaluation}. The selected reasoners were chosen for being freely
available, and include Pellet \cite{sirin2007pellet}, HermiT
\cite{shearer2008hermit}, FaCT++ \cite{tsarkov2006fact++} and TrOWL
\cite{TPR2010}.
